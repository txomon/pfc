\chapter{Dise�o de la integraci�n}

Este es un cap�tulo que tiene como objetivo explicar con un dise�o muy superficial un trabajo que se ha desarrollado m�s en el dise�o de bajo nivel, que no entra dentro del dise�o de c�digo, sino m�s bien en la l�nea de integraci�n con los est�ndares oficiales que pretende seguir este proyecto.

\section{Est�ndares de directorio}

Se pretende integrar Ksensor dentro de la estructura de directorios de una forma completa. Al igual que muchos otros drivers, o partes del kernel, esto se ha hecho creando una carpeta en el directorio que m�s raz�n de pertenecer ten�a. La idea principal detr�s de este directorio es introducir en esa parte todo lo desarrollado en el grupo de investigaci�n NQaS para su posible liberaci�n.

\section{Est�ndar de c�digo}

Tanto como es importante la integraci�n de los ficheros que se han desarrollado en NQaS, es importante acotar las modificaciones como est� hecho en el resto del kernel. Se han acotados todas las inserciones por categor�as o secciones, que permiten compilar ciertas funcionalidades, y otras no.

Todo esto debe ser asimismo configurable para facilitar el uso de las aplicaciones, por ello, se ha integrado todo en una categor�a que permite su total desactivaci�n, NQAS dentro de la secci�n de red, y dentro se han dividido las aplicaci�n Ksensor de los m�dulos auxiliares para validarlo, y dentro de la aplicaci�n Ksensor se pueden elegir opciones de depuraci�n y estad�sticas.

Es muy importante que el dise�o en esta parte concuerde con el c�digo y con la l�gica de la aplicaci�n, ya que dejar alg�n trozo suelto podr�a ser fatal y har�a que fallara la compilaci�n. En el plan de pruebas se especificar� como comprobar que el dise�o de esta secci�n ha sido correctamente implementado.

\section{Est�ndar de }