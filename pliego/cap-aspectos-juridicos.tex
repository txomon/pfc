\chapter{Aspectos jur�dicos}
Se contemplan los siguientes aspectos jur�dicos relacionados con la contrataci�n y desarrollo del proyecto.

\section{Fuerza mayor}
El contratista no ser� considerado responsable por el incumplimiento de las obligaciones en tanto la ejecuci�n de los trabajos se retrase o se hiciese imposible por causas de fuerza mayor. Son considerados causas de fuerza mayor todos aquellos sucesos o circunstancias, fuera del control del contratista o del comprador y cualesquiera otras circunstancias que fueran imprevisibles, o que siendo previsibles fueran inevitables, de acuerdo con la jurisprudencia y la doctrina legal sentada sobre este concepto en el C�digo Civil.

Las antedichas causas de fuerza mayor se tomar�n en consideraci�n �nicamente cuando afecten directamente al suministro.

\section{Arbitrajes y tribunales}

Tanto el comprador como el contratista se comprometen a cumplir las condiciones establecidas en el presente Pliego de Condiciones y en la documentaci�n complementaria del contrato, resolviendo por medio de acuerdos y negociaciones las posibles diferencias que puedan surgir entre ellos respecto a la aplicaci�n, desarrollo, cumplimiento, ejecuci�n o interpretaci�n de los mismos.

En caso de que cualquier posible discrepancia o controversia entre ellos no pudiese ser llevada a buen fin, resolvi�ndose con �xito en la forma anteriormente indicada, el contratista se compromete a someter tales diferencias a arbitraje, formalizado de acuerdo a las normas reguladoras del mismo, contenidas en la vigente Ley de Arbitraje de Derecho Privado.

El arbitraje se realizar� en Bilbao y en la escritura ha de figurar el compromiso del plazo de 30 d�as como t�rmino en que los �rbitros han de pronunciar el Laudo correspondiente.
El procedimiento de arbitraje estar� de acuerdo con la Legislaci�n Espa�ola y ser�
resuelto en derecho por tres �rbitros, uno elegido por cada parte y el otro ya designado.
Si las partes no llegaran a un acuerdo para designar el primer �rbitro, �ste ser�
seleccionado por insaculaci�n entre los tres �rbitros de una lista que haya sido
proporcionada por el Colegio de Abogados de Bilbao.
Los �rbitros proceder�n a liquidar los gastos del procedimiento arbitral, incluyendo sus
honorarios, y a establecer a cu�l de las partes tendr�n que cargarse los gastos de
enjuiciamiento.
Las partes declaran desde ahora aceptar las decisiones del Colegio Arbitral.
