\chapter{Condiciones legales y contractuales}

\section{Actas de Recepci�n}
La recepci�n provisional de los servicios se efectuar� una vez prestados los mismos, debiendo firmar ambas partes el Acta de Recepci�n Provisional. A partir de la creaci�n de dicha acta, comenzar� el plazo de reclamaci�n, cuya duraci�n ser� de dos semanas. Transcurrido este plazo, y a no ser que el contratista notifique los defectos para su subsanaci�n dentro del plazo, se suscribir� el Acta de Recepci�n Definitiva.

\section{Responsabilidades del cliente}
El cliente se responsabiliza del cumplimiento de toda normativa legal vigente relativa a los servicios contratados, incluyendo los permisos y licencias de utilizaci�n, as� como la normativa de propiedad intelectual del desarrollo realizado.

\section{Responsabilidad del proyectista}
El presente proyecto requiere de una implementaci�n en el kernel actual para que se cumplan las especificaciones sobre las que se ha planeado el dise�o de bajo nivel. En caso de haber cambios importantes en versiones posteriores, habr�a que volver a comprobar que todas las especificaciones son suficientes.

\section{Extinci�n del contrato}
El contrato se extinguir� bien por conclusi�n o cumplimiento del mismo, o por resoluci�n de una de las partes involucradas en el mismo. Ser�n consideradas causas de resoluci�n:
\begin{itemize}
\item El incumplimiento de las cl�usulas contenidas en el pliego de condiciones.
\item La extinci�n de la personalidad jur�dica de la sociedad mercantil de una de las partes, salvo que el patrimonio sea incorporado a otra entidad.
\item Mutuo acuerdo entre las partes.
\item La declaraci�n de quiebra o suspensi�n de pagos de una de las partes.
\end{itemize}

\section{Resoluci�n de conflictos}
Los litigios que puedan surgir sobre interpretaci�n o modificaci�n del contrato ser�n resueltos por los Juzgados y Tribunales, renunciando a cualquier otro fuera que pudiera corresponder a las partes. Por tanto, se encomienda a estos el nombramiento de �rbitros y la administraci�n del arbitraje cuyo Laudo las partes se obligan a aceptar.