\chapter{Paquetes de trabajo}
A continuaci�n se describen los paquetes de trabajo a desarrollar en el presente proyecto.

\begin{description}
\item[PT1] \textbf{Estudio de las necesidades del sistema.}\\
Se realiza un estudio del entorno en el que se va a utilizar el sistema y de los usuarios potenciales del mismo. El objetivo es que las especificaciones queden perfectamente definidas.

\item[PT2] \textbf{Estudio de las alternativas.}\\
En este paquete de trabajo se identifican partes del sistema para las que se presentan diferentes alternativas de dise�o. Se realiza un estudio de dichas alternativas para cada caso y se escogen las opciones adecuadas. Es totalmente necesario para no reimplementar funcionalidades existentes.

\item[PT3] \textbf{Dise�o de los m�dulos del proyecto.}\\
Teniendo en cuenta las especificaciones planteadas y las alternativas escogidas, en este paquete de trabajo se realiza el dise�o de los diferentes m�dulos que componen el proyecto.

\item[PT4] \textbf{Desarrollo de los m�dulos del proyecto.}\\
En este paquete de trabajo se implementan una serie de m�dulos prototipo cuyo objetivo es validar el dise�o de los m�dulos dise�ados en el paquete de trabajo anterior.

\item[PT5] \textbf{Integraci�n de los m�dulos, pruebas de validaci�n y de rendimiento.}\\
Se ejecutan los prototipos de los m�dulos para su validaci�n y se realiza un estudio de los resultados obtenidos para validar el dise�o de los m�dulos.

\item[PT6] \textbf{Gesti�n del proyecto.}\\
Este paquete de trabajo contiene las tareas administrativas y de gesti�n que se deben realizar a lo largo de todo el proyecto as� como la generaci�n de la documentaci�n del mismo.
\end{description}
