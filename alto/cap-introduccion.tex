\chapter{Introducci�n}

En este documento se describe el dise�o de alto nivel de la nueva implementaci�n de Ksensor. Primero se comenzar� por la descripci�n de la arquitectura general, posteriormente se explicar� la arquitectura de Ksensor y a continuaci�n se explicar�n los diferentes m�dulos que se van a a�adir al proyecto. Finalmente, se explicar�n la manera de integrar todos estos m�dulos como se ha descrito en el an�lisis de alternativas, dentro del SCVD Git, para acabar explicando la arquitectura del entorno de pruebas.

En primer lugar debe aclararse el alcance y los objetivos fijados para la realizaci�n de este proyecto. La finalidad de este proyecto es la \textit{Adaptaci�n y extensi�n de la herramienta de captura y an�lisis de tr�fico Ksensor}.

Al mencionar la adaptaci�n nos referimos a la migraci�n o actualizaci�n de la herramienta a una versi�n del kernel de Linux actual. Es importante remarcar que no se trata de redise�ar Ksensor al completo, sino m�s bien, hacer los cambios en el dise�o necesarios para su posible puesta en marcha en un kernel Linux actual. Tambi�n se pretende enfatizar que se quiere integrar todo el desarrollo de una manera uniforme y est�ndar en un repositorio paralelo al oficial.

En lo referente a extensi�n, se pretende ampliar no tanto la herramienta, si no lo dispuesto alrededor de la misma, para su validaci�n. Se pretende mejorar Ksensor modelando de forma te�rica el sistema para se�alar posibles cuellos de botella. Para ello, por ejemplo, se deben sacar estad�sticas del n�mero de interrupciones que ocurren durante un per�odo, relacion�ndolo con la tasa de saturaci�n de la red. Para conseguir todas estas mediciones, se pretende dotar al kernel de Linux de unos m�dulos ligeros, suponiendo ligero todo aquel m�dulo que no introduzca una carga de procesamiento importante en el sistema, que se encarguen de sacar los n�meros.

Por lo tanto, el orden que se va a seguir es la descripci�n de la arquitectura general del sistema a emplear, el kernel de Linux, seguidamente el m�dulo Ksensor y los m�dulos de traceo y estad�sticas, para acabar explicando la manera en la que se pretende integrar todo el trabajo de una manera uniforme.

Este documento pretende describir los m�dulos explicando sus interfaces y funcionamiento, as� como las decisiones que se han tomado en su dise�o. Para una descripci�n m�s detallada se puede acudir al documento de \textit{Dise�o de bajo nivel}.