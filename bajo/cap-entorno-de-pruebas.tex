\chapter{Entorno de pruebas}

En este cap�tulo, se describir�n los dos programas que permiten al entorno de pruebas utilizar los m�dulos desarrollados en este proyecto, el m�dulo de estad�sticas y el m�dulo de traceo. Se explicar� brevemente los comandos que se han utilizado, y cual es el orden que se sigue.

\section{M�dulo de estad�sticas}

El script, demonio, est� programado en Bash, ya que es un script trivial, que necesita utilizar de varios comandos del sistema, tales como modprobe, rmmod, sudo, etc. Primero se consigue el directorio de estad�sticas de la interfaz que nos interesa, en este caso, nos podemos aprovechar del fichero que crea Ksensor, \code{/proc/ksensor/iface} ya que en �l se describe cual es la interfaz de captura que interesa.

Esto se puede hacer con una l�nea en la que se lea el fichero, y se corte la salida \code{cat /proc/ksensor/iface | cut -c1-4}. Despu�s en caso de recibir el par�metro start, deber� cargar el m�dulo, con \code{modprobe times}, y despu�s resetear las estad�sticas, con un \code{echo 1> /proc/capture\_stats/reset}.

En caso de recibir el par�metro stop, deber�n sacarse todas las estad�sticas de la interfaz, por ejemplo, utilizando el comando \code{grep -G -R ".*" /proc/capture\_stats/\$iface} se puede conseguir sacar todos los parametros separados por dos puntos. Adem�s este deber� descargar el m�dulo del sistema, utilizando \code{rmmod capture\_stats}.

\section{M�dulo de trazas}

La arquitectura actual no est� preparada para pasar un volumen tan grande de trazas, como por ejemplo las que salen utilizando este sistema para tracear las softirqs. Por ello, en vez de sacar las trazas por la arquitectura, se sacar�n las trazas a un fichero local, y utilizando la fecha y hora de las mismas.

Para esto, se carga el m�dulo con \code{modprobe} cuando as� se lo indica el demonio, despu�s, antes de sacar el m�dulo, se descarga el contenido del fichero con un \code{cat /proc/capture\_tracer > fichero\_temporal}, coloc�ndolo en una carpeta con la fecha de la prueba asociada.