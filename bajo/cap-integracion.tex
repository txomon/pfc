\chapter{Dise�o de la integraci�n}

En este cap�tulo se va a explicar como se integran los diferentes m�dulos entre s� y con el kernel original. Teniendo en consideraci�n el diagrama de este apartado en el dise�o de alto nivel, se ha planeado dicha integraci�n como una serie de par�metros, dependientes unos de otros, de manera que dependiendo lo que se especifique, se puede elegir los m�dulos que se van a compilar, y con todas las funcionalidades posibles.

Aunque esta secci�n habla de la integraci�n existente de los m�dulos, se va a evitar explicar lo que compete a la parte de estandarizaci�n. Por lo tanto, se especificar� la manera en la que estos m�dulos activan funcionalidades en otros, pero se dejar� de lado la forma en la que se fusionan.

\section{M�dulo de estad�sticas y Ksensor}
El m�dulo de estad�sticas recoge datos de tiempos que tarda Ksensor, tanto en encolar paquetes como en sacarlos de la cola. La manera de hacerlo es simple:

\begin{lstlisting}[caption={Estad�stica de medici�n del desencolado de un paquete}, label={stats-dequeue-ksensor}]
struct sk_buff *dequeue_skb_ksensor(void)
{
	unsigned long long int start_time;
	
	start_time=sched_clock();
	
	/* ... */
	
	skb->napi->capture_stats.dequeue_time += (sched_clock() - start_time);
	return skb;
}
\end{lstlisting}

Como se puede observar, en este caso, el par�metro que se ha utilizado para guardar la medida, napi, se ha a�adido, ya que al haber hecho las mejoras de Ksensor para su adaptaci�n a dispositivos multicolas, la cola de la que se ha extra�do el paquete tambi�n es importante.

Esto se ha hecho a�adiendo un puntero a estructura \code{struct napi\_struct} en el descriptor de buffer de paquete \code{struct sk\_buff}.

\begin{lstlisting}[caption={Modificaci�n a la estructura sk\_buff para filtrado por cola}, label={stats-skbuff}]
struct sk_buff {
	/* These two members must be first. */
	struct sk_buff		*next;
	struct sk_buff		*prev;

	ktime_t			tstamp;

	struct sock		*sk;
	struct net_device	*dev;
	struct napi_struct	*napi;

	/* ... */
}
\end{lstlisting}

\section{M�dulo de estad�sticas y m�dulo de trazas}

Aunque el como se ha explicado antes el m�dulo de trazas es totalmente independiente del resto de las aplicaciones de NQaS, si que comparte mediciones en las trazas actuales con el m�dulo de estad�sticas. Por ello, las mediciones realizadas con \code{sched\_clock} han sido dispuestas en variables a parte.

Como ejemplo, se comparte la medida de tiempos en la softirq, \code{net\_rx\_action}, de la siguiente manera

\begin{lstlisting}[caption={Integraci�n entre el m�dulo de estad�sticas y trazas}, label={stats-tracer-integration}]
static void net_rx_action(struct softirq_action *h)
{
#if IS_ENABLED(CONFIG_NQAS_CAPTURE_TRACER) ||IS_ENABLED(CONFIG_NQAS_STATS)
	unsigned long long int time_start;
#endif /* CONFIG_NQAS_CAPTURE_TRACER || CONFIG_NQAS_STATS */
#if IS_ENABLED(CONFIG_NQAS_CAPTURE_TRACER)
	struct capture_tracer_measure capture_tracer_measure = {0,0,netdev_budget,0};
#endif  /* CONFIG_NQAS_CAPTURE_TRACER */

#if IS_ENABLED(CONFIG_NQAS_CAPTURE_TRACER) ||IS_ENABLED(CONFIG_NQAS_STATS)
	time_start=sched_clock();
#endif /* CONFIG_NQAS_CAPTURE_TRACER || CONFIG_NQAS_STATS */

	/* ... */

#if IS_ENABLED(CONFIG_NQAS_CAPTURE_TRACER) ||IS_ENABLED(CONFIG_NQAS_STATS)
	time_end = sched_clock();
#endif /* CONFIG_NQAS_CAPTURE_TRACER || CONFIG_NQAS_STATS */
#if IS_ENABLED(CONFIG_NQAS_STATS)
	list_for_each_entry(n, &sd->polled_list, polled_list) {
		n->capture_stats.softirq_ac += time_end - time_start;
		n->capture_stats.softirq_num++;
	}
#endif /* CONFIG_NQAS_STATS */

#if IS_ENABLED(CONFIG_NQAS_CAPTURE_TRACER)
	capture_tracer_measure.end = time_end;
	capture_tracer_measure.packets = netdev_budget - budget;
	if(save_capture_tracer_measure != NULL)
		save_capture_tracer_measure(&capture_tracer_measure);
#endif /* CONFIG_NQAS_CAPTURE_TRACER */
	return;
}
\end{lstlisting}

En este caso, nos aprovechamos de que el compilador dependiendo de las etiquetas que compila, eliminar� las variables que no tengan sentido, como por ejemplo, si estamos compilando las estad�sticas, eliminar� la variable \code{time\_end}, e incluso seguramente, elimine tambi�n \code{time\_start}.

Aunque se hayan mostrado los flags de compilaci�n en este trozo de c�digo, corresponde a la parte de estandarizaci�n explicar su uso y configuraci�n.