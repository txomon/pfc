% !TeX spellcheck = es_ES
\documentclass[a4paper]{report} % Para tener un documento grande pero sin ser un libro (solo hay esta posibilidad)
\usepackage[latin1]{inputenc}   % Para la codificaci�n
\usepackage{textcomp}           % Para los simbolos de euro
\usepackage[spanish]{babel}     % Para el lenguaje que tenga bien
\usepackage[T1]{fontenc}        %      los saltos de linea con gui�n etc.
\bibliographystyle{alpha}       % Estilo de bibliograf�a
\usepackage{makeidx}            % Hacer �ndices
\usepackage{fancyhdr}           % Cambiar las cabeceras del documento
\usepackage{graphicx}           % Para insertar im�genes
\graphicspath{{./images/}}      % La ruta relativa de las im�genes
\usepackage{hyperref}           % Hacer links guays entre secciones y hacia fuera
\usepackage[hypcap]{caption}    % Hacer que las referencias se vean, y no se queden por encima (referenciando imagenes, etc.)
\hypersetup{
	colorlinks=true,
	linkcolor=blue,
	citecolor=red,
}
\usepackage[table]{xcolor}      % Dar colores a las tablas
\usepackage{eurosym}            % para tener el simbolo del euro


% % % % % % % % % % % % % % % % % % % % % % %
% Secci�n de comandos personalizados        %
% % % % % % % % % % % % % % % % % % % % % % %
\newcommand{\reference}[2]{\hyperref[#1]{\textit{\ref*{#1} #2}}}

\title{Adaptaci�n y extensi�n de la herramienta de captura y an�lisis de tr�fico Ksensor}
\author{Javier Domingo Cansino}
\makeindex

\begin{document}
\pagenumbering{Roman}
\maketitle % Crear el t�tulo
\tableofcontents % Crear la tabla de contenidos

\pagenumbering{arabic}
% !TeX spellcheck = es_ES
% !TeX root = main.tex

\chapter{Introducci�n}

En los �ltimos a�os, las tecnolog�as inform�ticas de la comunicaci�n est�n revolucionando la manera de comunicarse del mundo entero. Tener acceso a Internet se ha convertido en algo necesario para poder ponerse en contacto con el resto del mundo.

Cada vez hay m�s dispositivos conectados, con el consiguiente crecimiento del tr�fico en todas las redes. Incluso empresas que antes contrataban redes privadas para sus comunicaciones, ahora utilizan Internet para interconectar sus redes.

Como en cualquier servicio cr�tico el mantenimiento debe ser proactivo y por ello se han creado m�todos para garantizar la m�xima eficiencia de las comunicaciones. Para ello se debe garantizar la seguridad en la red y la detecci�n instant�nea de problemas, asegurando as� la \textit{QoS} (\textit{Quality of Service}, Calidad de Servicio).

Para la securizaci�n de la red se implantan \textit{firewalls}, \textit{proxys}, e \textit{IDS}s (\textit{Intrusion Detection System}, Sistema de Detecci�n de Intrusi�n), que se encargan de evitar accesos externos no autorizados a la red, controlar el tr�fico saliente, y detectar posibles atacantes dentro de la red.

El m�todo para asegurar la calidad de las comunicaciones es la medici�n de par�metros de \textit{QoS}, tales como retardo de los paquetes, errores en la transmisi�n, tipos de tr�fico comunes, rutas �ptimas, etc.

Son varias las soluciones comerciales que implementan estos servicios pero actualmente no hay ninguna que implemente todo en un mismo sistema. Adem�s, no hay ning�n producto de c�digo libre que aproveche al m�ximo la posible eficiencia del equipo, haciendo un an�lisis \textit{online}\footnote{Online se refiere a hacer el an�lisis mientras se captura, en vez de capturar, guardar y luego analizar.}.

En el grupo de investigaci�n \textit{Network Quality and Security (NQAS)} se ha dise�ado un sonda que implementa algunas de las citadas funcionalidades. En el dise�o de la misma se ha contemplado la posibilidad de m�s adelante poder ampliarlo de una manera sencilla. Como valor a�adido fundamental, aprovecha al m�ximo los recursos disponibles, ya que ha sido programado de una manera que permita ejecuci�n multi-hilo.

Este prototipo se llama ksensor\cite{KABO05}, aunque no es el primero que se ha creado. La l�nea de investigaci�n principal, llamada hi-sensor, ha ido haciendo cada vez m�s complejo el dise�o y la programaci�n del prototipo con el objetivo de mejorar la eficacia.

A grandes rasgos, primero se dise�� de forma que la aplicaci�n fuera un programa de usuario que capturaba a trav�s de una librer�a est�ndar para la captura de paquetes de forma promiscua. Se utilizaron tambi�n las librer�as para tener computaci�n multi-hilo, y se consiguieron unos resultados prometedores.

Cuando se empezaron a analizar redes de alta velocidad en saturaci�n se vio que hab�a un problema: El equipo estaba continuamente capturando, y no hab�a espacio para el an�lisis. Para solucionar este problema, se migr� a espacio de kernel.

El kernel es la base que hace que todos los programas funcionen. Es un programa que se encarga de gestionar los dispositivos y recursos del ordenador para ofrecer una interfaz libre de detalles a los programas. Adem�s, tambi�n se encarga de hacer que varios programas se ejecuten de una manera que proporcione a todos y cada uno la forma de parecer que est�n en ejecuci�n continua. A parte de eso, se encarga de que los programas sean capaces de hacer un direccionamiento virtual de una forma transparente a ellos, manejando eficazmente una MMU (\textit{Memory Management Unit}, Unidad de Gesti�n de Memoria), de manera que permite que estos sea posible compilarlos como si solo fueran a existir ellos.

En general, del dise�o de un buen kernel depender� la eficiencia del sistema. En nuestro caso, el dise�o del kernel no est� hecho para nuestro tipo de sistema, una sonda de tr�fico, pero a trav�s de una serie de modificaciones, podemos cambiar el comportamiento del sistema para amoldarlo a nuestro caso de uso. Todo sea dicho, la eficiencia del kernel actual es buena para los sistemas operativos de car�cter general, en los que capturar los paquetes de uno mismo es suficiente, y que es una cantidad peque�a en comparaci�n con el tr�fico de la red.

El cometido del sistema que queremos es �nicamente para capturar y procesar paquetes y por lo tanto, la m�xima eficiencia se puede describir como el equilibrio entre el tiempo en el que el ordenador esta capturando y el que est� analizando. Se tienen que capturar exactamente el n�mero de paquetes que se van a analizar.

La implementaci�n actual del sistema ha quedado obsoleta para kernels actuales, ya que el tratamiento de los paquetes se hace de una forma diferente. Ahora se crean \textit{superpaquetes} ensamblando desde la captura los paquetes que tienen una serie de par�metros iguales, como pueden ser la IP de origen, IP de destino, puerto TCP destino y origen. De esta manera, al pasar un �nico paquete (aunque grande) a la pila de protocolos, se acelera su tratamiento, ya que las copias de los paquetes se pueden hacer en una �nica transacci�n.

Por lo tanto, el proyecto se enfoca a mejorar el anterior dise�o para adaptarlo a la nueva manera de hacer las cosas, crear una serie de herramientas para validar el dise�o y adaptar el proyecto a los est�ndares defacto existentes en kernel.org para hacer posible la liberaci�n p�blica del c�digo. La tarea de mayor relevancia en estos momentos es posibilitar un estudio comparativo entre un sistema normal y la aplicaci�n.
% !TeX spellcheck = es_ES
% !TeX root = main.tex

\chapter{Objetivos}
El objetivo principal del proyecto es la \textbf{adaptaci�n del programa ksensor y modificaciones asociadas en kernel a una versi�n actual y el desarrollo de herramientas para la validaci�n del prototipo}. El dise�o actual es el descrito en el proyecto KSENSOR.

Por lo tanto, los objetivos a conseguir mediante este proyecto son:

\begin{enumerate}
\item Captura eficiente del tr�fico de red
\item Paralelizaci�n del an�lisis
\item Compatibilidad con los m�dulos externos al sensor
\item Estudio comparativo de rendimientos
\end{enumerate}

\section{Captura eficiente del tr�fico de red}
Ksensor debe optimizar el proceso de captura de los paquetes, eliminando si es preciso acciones realizadas por el kernel que no son estrictamente necesarias para el buen funcionamiento del sistema (recolecci�n de estad�sticas, copia del paquete a espacio de usuario, etc.).

El kernel de Linux es de prop�sito general, por lo que no est� espec�ficamente dise�ado para realizar una captura eficiente del tr�fico. Ya que la mayor parte de los sistemas de an�lisis de tr�fico a nivel de usuario se apoyan para realizar la captura en las facilidades proporcionadas por sistemas operativos de prop�sito general, su rendimiento no es �ptimo. El sensor que se propone en este proyecto debe, al trabajar directamente desde el nivel de kernel, resolver dichas ineficiencias, evitando por ejemplo capturar paquetes que posteriormente no pueda analizar, lo cual sin duda degradar�a el rendimiento.

\section{Paralelizaci�n del an�lisis}
El an�lisis del tr�fico en redes de alta velocidad es un proceso costoso, que le exige al sensor un alto rendimiento y una capacidad de procesamiento que no se pueden conseguir con las arquitecturas convencionales. Con el fin de acomodar el sensor a las necesidades de procesamiento requeridas por la red, en este proyecto se propone mejorar las prestaciones de una m�quina aumentando el n�mero de procesadores.

As� pues, con un dise�o en base a plataformas multiprocesador, se podr� disponer de m�ltiples instancias de an�lisis en el sensor, trabajando de forma concurrente sobre el tr�fico capturado. El sistema dise�ado deber� permitir, en la medida de lo posible, el paralelismo en el procesado de los paquetes, a fin de obtener un alto rendimiento. Se deber� resolver, tambi�n, la sincronizaci�n en el acceso a la informaci�n compartida entre instancias.

\section{Compatibilidad con los m�dulos externos al sensor}
En el presente proyecto se propone migrar el sensor a nivel de kernel con el fin de optimizar su rendimiento. Debemos tener en cuenta, sin embargo, que el sensor es tan s�lo el n�cleo del sistema de an�lisis de tr�fico, y que para su buen funcionamiento necesita trabajar de forma coordinada con el resto de componentes del sistema, en particular, con el parser y el m�dulo de procesamiento offline [DEL03], los cuales necesitan acceder al mapa de memoria del sensor para realizar su cometido. Por esta raz�n, el sensor debe ofrecer una interfaz unificada y gen�rica para el acceso a su mapa de memoria desde el espacio de usuario, y controlar el acceso al mismo a trav�s de alg�n mecanismo de sincronizaci�n. El dise�o que se realice deber� contemplar adem�s la posibilidad de que existan nuevos m�dulos en el futuro.

\section{Estudio comparativo de rendimientos}
El sensor a nivel de kernel (ksensor) pretende resolver las limitaciones inherentes al dise�o del sensor en espacio de usuario (Adviser), con el fin de obtener una mejora de prestaciones que posibiliten el an�lisis del tr�fico en redes de alta velocidad. Sin embargo, una implementaci�n deficiente o, quiz�s, la existencia de factores que no se han considerado podr�an impedir obtener los resultados esperados. Por lo tanto, se hace necesario poder comprobar y cuantificar las mejoras de rendimiento obtenidas, para poder valorar la viabilidad de un futuro producto comercial.

En este proyecto se implementar� un prototipo de ksensor, el cual permitir� realizar las pruebas de rendimiento pertinentes, con el fin de poder valorar los resultados y comprobar las mejoras obtenidas, si las hubiera. Las pruebas de rendimiento permitir�n, asimismo, conocer los par�metros de configuraci�n �ptimos en el sensor.

% !TeX spellcheck = es_ES
% !TeX root = main.tex

\chapter{Beneficios}
En esta secci�n se presentan los principales beneficios que aporta el proyecto, los cuales se engloban fundamentalmente en tres campos:
\begin{enumerate}
\item Beneficios t�cnicos
\item Beneficios econ�micos
\item Beneficios sociales
\end{enumerate}

\section{Beneficios t�cnicos}
La ejecuci�n del proyecto planteado aporta una serie de beneficios t�cnicos en el campo del an�lisis de tr�fico de red. A continuaci�n, se describe individualmente cada uno de ellos.

\subsection{An�lisis del tr�fico en redes de alta velocidad}
Optimizando el proceso de captura de los paquetes, el sensor a nivel de kernel permitir� realizar el an�lisis del tr�fico en redes de alta velocidad. �ste es el principal objetivo que se ha definido en el proyecto, as� como el principal beneficio t�cnico que se espera obtener. 
El an�lisis del tr�fico en redes de alta velocidad exige un coste computacional elevado, que no es posible obtener mediante las arquitecturas cl�sicas. Dicho problema se encuentra actualmente en fase de investigaci�n y es objeto de estudio por la comunidad cient�fica. 
El proyecto que se plantea propone una soluci�n software que, al funcionar desde el nivel de kernel, permitir� efectuar una captura eficiente de los paquetes, y al trabajar sobre plataformas multiprocesador, realizar el an�lisis concurrente de los mismos.

\subsection{Integraci�n de tipos de an�lisis}
Ksensor ha sido dise�ado de forma que pueda realizar cualquier tipo de an�lisis sobre el tr�fico capturado. Cuando se captura un paquete de la red, la sonda lo procesa atendiendo a la l�gica de procesamiento cargada en memoria por el parser. Dicho mapa de memoria se organiza en la forma de un �rbol de nodos de decisi�n, que el motor de an�lisis del sensor recorre para determinar el tratamiento que se le ha de dar al paquete.

El dise�o de la l�gica de nodos es gen�rico y multidisciplinar, por lo que permite efectuar cualquier tipo de tratamiento sobre el tr�fico capturado. As� pues, el sensor es v�lido en diferentes �mbitos de la telem�tica como la detecci�n de intrusi�n, la monitorizaci�n inteligente de red o la medida de par�metros de calidad de servicio.

\subsection{Arquitectura de pruebas}
Con el fin de poder realizar pruebas de rendimiento en el sensor de forma automatizada, en este proyecto se ha desarrollado una arquitectura de pruebas, basada en la plataforma gen�rica dise�ada en \cite{AABS05}. Esta arquitectura permite comprobar el comportamiento tanto de Adviser como de Ksensor cuando se los somete a un patr�n de tr�fico determinado. Para ello se utilizan plantillas de pruebas o tests, que el sistema ejecuta de forma automatizada y sucesiva, configurando la sonda y los inyectores seg�n se especifique en la plantilla.

En la arquitectura actual, se dispone de inyectores hardware, llamados DAG, que son capaces de inyectar a una tasa capaz de saturar el enlace, permitiendo probar el sistema bajo condiciones de estr�s, que se dan en redes de alta velocidad. Asimismo, la arquitectura recoge diversas estad�sticas de las pruebas realizadas, de cada uno de los componentes que participan en las mismas: switch, interfaces de red, inyectores, sonda, etc.


\section{Beneficios econ�micos}
La utilizaci�n del sistema dise�ado como herramienta de an�lisis de tr�fico introduce un conjunto de beneficios econ�micos en relaci�n con las aportaciones t�cnicas que se han expuesto anteriormente.

\subsection{An�lisis del tr�fico en redes de alta velocidad}
En la actualidad no existen apenas herramientas en el mercado capaces de realizar el an�lisis del tr�fico en redes de alta velocidad. Ksensor ha sido espec�ficamente dise�ado para satisfacer las necesidades de captura y procesamiento t�picas en estas redes. As�, un producto derivado del prototipo de ksensor podr�a ser utilizado en los segmentos de alta velocidad de los Proveedores de Servicios de Internet (ISP), en los cuales podr�a resultar �til la medida de par�metros de QoS o la monitorizaci�n del tr�fico cursante. Podr�a ser utilizado, tambi�n, como herramienta de an�lisis en las redes de �rea local de las empresas.

\subsubsection{Localizaci�n de fallos de red}
El buen funcionamiento de las redes corporativas es indispensable en el ejercicio empresarial. Esto es especialmente cierto en la actualidad, en que la mayor parte de las transacciones se realizan de forma telem�tica. Si por alg�n motivo la red completa cae puede significar un par�n en el proceso productivo. La indisponibilidad del servicio impide que los clientes puedan contactar con la empresa o que se confirmen los pedidos. En cualquier caso, afectar� negativamente a la producci�n y, en periodos de indisponibilidad largos, degradar� su imagen.

Haciendo uso de la sonda se podr�a reducir el tiempo entre que ocurre el fallo y se arregla. La sonda analizar�a todo el tr�fico que circula por la red y podr�a programarse por ejemplo para monitorizar las conexiones con el servidor web. Si �ste fallara, la sonda lo detectar�a y podr�a generar una alarma para indicarlo. De este modo, el administrador de la red podr�a utilizar la sonda como herramienta de diagn�stico para localizar el origen del problema.

En un caso de uso m�s avanzado, se podr�a incluso llegar a detectar los fallos, sin llegar a caer el servicio. Por ejemplo, se podr�a programar la sonda para que analizara las retransmisiones, el retardo y par�metros de rutado de paquetes, resultando en que en caso de detectarse un n�mero de paquetes con el TTL a punto de expirar, demasiadas retransmisiones o cualquier otro s�ntoma de inestabilidad en la red, avisara al administrador, o incluso, activara directivas de balanceo de tr�fico.

\subsubsection{Detecci�n de ataques en la red}
La seguridad es uno de los factores clave en el dise�o de las redes corporativas. De no preservarse la integridad de las comunicaciones, personas ajenas a la empresa podr�an obtener informaci�n privada o amenazar de alg�n modo su funcionamiento normal. Si bien ya existen dispositivos espec�ficos con este fin, como los firewall, debe tenerse en cuenta que gran parte de los ataques proceden del interior de la empresa, y que por tanto pueden burlar sin problemas las medidas de seguridad perimetral. La sonda, mediante el an�lisis del tr�fico de la red, podr�a realizar la labor de un sistema de detecci�n de intrusi�n, buscando patrones de tr�fico que considera sospechosos, independientemente de su procedencia, y por tanto �til tambi�n en estos casos.

\subsubsection{Reducci�n de los dispositivos necesarios}
El dise�o de ksensor permite integrar en un �nico dispositivo diferentes tipos de an�lisis, que de lo contrario �nicamente podr�an realizarse utilizando un conjunto de
sistemas espec�ficos. As�, la sonda puede obtener diversos par�metros de la red, monitorizar las conexiones o incluso buscar patrones de tr�fico al igual que en los sistemas de detecci�n de intrusi�n.


\section{Beneficios sociales}

Los beneficios sociales que aporta la sonda se derivan principalmente de la utilizaci�n de la arquitectura automatizada de pruebas, expuesta anteriormente en el apartado de beneficios t�cnicos.

\subsection{B�squeda de par�metros �ptimos en la sonda}
La arquitectura de pruebas puede utilizarse para comprobar el funcionamiento de la sonda con diferentes par�metros de configuraci�n, sometido a diversas cargas de red que tambi�n es posible especificar. De este modo, se pueden obtener de forma emp�rica los par�metros �ptimos configurables en el sensor, como puede ser, por ejemplo, el tama�o de la cola en la que se almacenan los paquetes capturados, el n�mero �ptimo de tarjetas de red, o la frecuencia de captura de paquetes.

\subsection{Comparativa de rendimientos entre sondas}
La arquitectura de pruebas se puede utilizar para realizar una comparativa de rendimiento entre las sondas a nivel de kernel ksensor y su nueva implementaci�n actualizada, con la finalidad de apreciar el comportamiento que �stos presentan cuando se los somete a un patr�n de tr�fico determinado. Tal y como la ejecuci�n de dichas pruebas permiti�, asimismo, justificar la realizaci�n de este proyecto de continuaci�n, al haberse obtenido resultados favorables en el rendimiento de ksensor.

\subsection{Reducci�n del tiempo invertido en pruebas}
Las pruebas se realizan de forma automatizada, con lo cual �nicamente es necesario definir una bater�a de pruebas y el sistema las ejecutar� sucesivamente en el orden que se hayan especificado. En consecuencia, el tiempo que una persona ha de invertir en realizar las pruebas se reduce notablemente y por lo tanto reduciendo costes.

% !TeX spellcheck = es_ES
% !TeX root = main.tex

\chapter{An�lisis de alternativas}
En este cap�tulo se expondr�n primeramente las especificaciones y requerimientos b�sicos del presente proyecto, describiendo brevemente la arquitectura inicial de la cual se parte en el dise�o de la sonda e indicando los requerimientos que se le imponen inicialmente.

Se expondr�n asimismo las mejoras que se pretenden conseguir con el nuevo dise�o, indicando los requisitos que debe cumplir el kernel del sistema operativo para llevarlas a cabo. A continuaci�n, se expondr�n las alternativas de dise�o que se deben considerar en la realizaci�n del proyecto y se har� un estudio para determinar cu�l es la �ptima en cada caso.

A lo largo del dise�o se han de tomar decisiones que van a contribuir de forma decisiva al �xito o fracaso del proyecto. Por esta raz�n, conviene analizar minuciosamente las alternativas que existen. No es necesario tener en cuenta todas y cada una de las posibilidades pero s� las que puedan resultar cr�ticas a la hora de aportar una soluci�n.

As� pues, se partir� de los requerimientos apuntados en el primer apartado para a continuaci�n plantear las alternativas concernientes a los puntos m�s cr�ticos del dise�o, y una vez tomadas las decisiones oportunas, se pasar�, en el cap�tulo de Dise�o, a describir con mayor detenimiento la soluci�n general propuesta, incidiendo especialmente en las partes previamente discutidas.

\section{Especificaciones y requerimientos b�sicos}
\label{sec:especificaciones-y-requerimientos-b�sicos}
En el presente proyecto se plantea el desarrollo de una sonda de an�lisis de tr�fico a nivel de kernel, para lo cual se parte de un dise�o previo de la sonda (Ksensor), realizado en un proyecto anterior \cite{KABO05}. La sonda es el n�cleo de una arquitectura dedicada al an�lisis de tr�fico, la cual incluye adem�s de la sonda, algunos m�dulos externos a �ste pero que tienen relaci�n directa con �l, como el parser o el OPM. As� pues, se hace necesario considerar los requerimientos que imponen estos m�dulos, ya que pueden condicionar en buena parte el dise�o que se realice en el nivel de kernel.

Por otro lado, conviene especificar las mejoras de dise�o que se pretenden obtener con la migraci�n de Ksensor, en base a las cuales se ha justificado la realizaci�n de este proyecto. El cumplimiento o no de dichas mejoras condicionar�n el �xito o fracaso del proyecto, por lo que se plantear�n tambi�n las mejoras que ha tenido el kernel de Linux para poder obtenerlas.

\subsection{Arquitectura de partida}
En la siguiente figura, se puede ver la arquitectura actual del sistema implementado Ksensor. Aunque los m�dulos que componen el sistema vienen de un sistema anterior llamado Adviser, se explicar�n brevemente a continuaci�n.

\begin{figure}[h]
\centering
\includegraphics[width=\textwidth]{arquitectura-previa}
\caption{Esquema de la arquitectura previa de ksensor}
\label{fig:arquitectura-previa}
\end{figure}

En la figura de la \reference{fig:arquitectura-previa}{arquitectura previa} se pueden observar los m�dulos que componen el sistema. Los paquetes se capturan en las tarjetas de red y el m�dulo de captura se encarga de procesar directamente los paquetes, sirvi�ndolos cuando se solicita a las instancias del m�dulo de procesamiento.

La sonda es el n�cleo del sistema de an�lisis. El resto de m�dulos pretenden bien configurar su comportamiento (parser) o bien recoger los resultados del procesamiento realizado (OPM). En cualquier caso, el nexo de uni�n entre todos ellos es la memoria interna del sensor, a la que tambi�n nos referiremos como mapa de memoria. En el mapa de memoria se almacena la l�gica de decisi�n que posteriormente utilizar� tanto en el m�dulo de captura, para decidir si un paquete debe ser capturado, como en el m�dulo de procesamiento del sensor para determinar el an�lisis que se ha de realizar cada vez que se captura un paquete.

El parser es el encargado de cargar en la memoria del sensor la informaci�n que �ste requiere para funcionar de forma correcta. Para ello, el parser toma como
entrada un fichero de reglas en XML especificado por el administrador del sistema. El fichero de reglas se escribe utilizando un lenguaje propio, el cual se ha definido teniendo en cuenta muchas caracter�sticas de entornos de captura de tr�fico como sistemas de detecci�n de intrusi�n, cortafuegos, etc.

El parser lee e interpreta el fichero de reglas que haya especificado el administrador del sistema y lo traduce al formato que entiende el sensor, almacenando en el mapa de memoria la l�gica adecuada y la informaci�n de entorno necesaria para el an�lisis de tr�fico. Para ello, se crean varias listas enlazadas: una para el �rbol de nodos, otra para los tipos de datos, acciones peri�dicas, etc. Tambi�n se reserva un espacio para almacenar las variables, que ser� donde el sensor deposite la informaci�n y estad�sticas que se le han pedido.

Si el parser es la interfaz de entrada al sensor, el m�dulo de procesamiento offline (OPM) constituye su interfaz de salida. El m�dulo de procesamiento offline es el encargado de recolectar toda la informaci�n que se obtiene como resultado del an�lisis del tr�fico y de formatearla para su posterior procesamiento fuera del sensor.

En el nuevo dise�o se va a dejar este dise�o de los m�dulos, ya que se ha comprobado emp�ricamente que no plantea ning�n tipo de degradaci�n de rendimiento, y adem�s, permite que la aplicaci�n Ksensor pueda estar en ejecuci�n o no, haciendo uso de la propiedad de modularidad del kernel de Linux.

\subsection{Mejoras del dise�o}
Este proyecto propone una adaptaci�n del dise�o existente de Ksensor a una versi�n de Linux actual. Preservando su compatibilidad con las herramientas externas pero mejorando el funcionamiento interno, la integraci�n en el �rbol de kernel, y creando nuevas herramientas que permitan estudiar el comportamiento del sistema, tanto con Ksensor funcionando como deshabilitado.

Con un an�lisis exhaustivo del c�digo de las rutinas de recepci�n de paquetes, se ha llegado a la conclusi�n de que a�n con el sistema en saturaci�n, hay recursos que se destinan a otros procesos de red, como el env�o de paquetes. Por ello, es necesario, no tanto cambiar el dise�o del sensor, si no crear las herramientas que permitan tener acotados los procesos que no nos interesan para poder planificar nuestro sistema con los par�metros �ptimos, en vez de hacerlo a prueba y error.

Tambi�n se pretende mejorar el rendimiento de captura, adaptando el n�mero de paquetes a capturar a las necesidades actuales del sistema, ya que as� se evita que la cola var�e de tama�o en forma de dientes de sierra, o que se intenten capturar paquetes que no deber�an entrar en la cola. En el apartado de dise�o se ampliar�n m�s estos conceptos.

\subsection{Requerimientos de la migraci�n}
Uno de los mayores prop�sitos de la migraci�n es aprovechar al m�ximo todas las herramientas disponibles para la ejecuci�n del proyecto, y para facilitar su desarrollo. Concretamente, se van a utilizar las herramientas que utilizan los desarrolladores de kernel, tales como un sistema de control de versiones, adaptando la manera en la que se tiene estructurado el c�digo para una �ptima utilizaci�n de las herramientas. Tambi�n se va a establecer una forma uniforme de integraci�n, ya que no ser�n solo mejoras a ksensor, si no tambi�n utilidades que pueden tener valor a la hora de diagnosticar sistemas.

\section{Estudio de alternativas de dise�o}
En este punto se estudian las diferentes alternativas de migraci�n que se tendr�n que considerar en la ejecuci�n del proyecto. En una primera aproximaci�n, se puede observar que muchas de estas alternativas tienen dependencias entre s�, por lo que las decisiones que se tomen para una podr�n influir de alguna manera sobre las siguientes.

Para cada alternativa se establecer�n diversos criterios que nos permitir�n seleccionar la opci�n m�s apropiada. Se tendr�n en cuenta tambi�n los requerimientos iniciales se�alados en el apartado \reference{sec:especificaciones-y-requerimientos-b�sicos}{Especificaciones y requerimientos b�sicos}.

\subsection{Sistema de control de versiones}
En primer lugar, debe decidirse cual es el sistema de control de versiones m�s adecuado para el proyecto. Esta decisi�n es muy importante porque har� depender en gran medida las posibilidades de integraci�n con el kernel oficial.

La herramienta electa debe ser eficaz para tama�os de repositorios grandes, flexibilidad de trabajo, opciones de sincronizaci�n con otros repositorios. Estas necesidades se especificar�n en la secci�n \reference{scv-criterios}{Criterios de selecci�n}. Es necesario por lo tanto aclarar algunos t�rminos que se van a utilizar en los siguientes puntos con regularidad.

\begin{description}
\item[Sistema de control de versiones] 
Un SCV o VCS por sus siglas en ingl�s, es una herramienta que facilita a los desarrolladores tener un control de cambios sobre sus proyectos. Se toma un directorio como ra�z, y se ordena el proyecto en ella. Cada vez que un desarrollador desea guardar, ejecuta un comando o presiona un bot�n a trav�s de una interfaz gr�fica y se guarda en el repositorio, de tal manera que queda accesible para poder volver a esa versi�n en cualquier momento.

Este tipo de sistemas posibilitan que en proyectos grandes se pueda hacer un an�lisis forense de los momentos en los que se han introducido cambios que pod�an contener alg�n tipo de regresi�n o posibiliten marcar algunas versiones como estables para su publicaci�n.

\item[SCV Centralizado o Distribuido]
Hay dos grandes grupos de SCVs, en los que la filosof�a de uso y trabajo es muy diferente.

En los centralizados (SCVC), hay un servidor central en el que se hace toda la gesti�n de versiones. El desarrollador trabaja en su equipo y cuando decide guardar, manda al servidor central la versi�n, que introduce las versiones de todos los desarrolladores en un �nico repositorio.

En los distribuidos (SCVD), el servidor es el propio repositorio local, en el que el desarrollador, guarda su trabajo sin necesidad de estar conectado a ning�n sitio, estos sistemas est�n llenos de herramientas para la sincronizaci�n de versiones entre diferentes repositorios.

\end{description}

\subsubsection{Estudio de las alternativas}
A continuaci�n se tendr�n en consideraci�n los sistemas de control de versiones m�s relevantes y usados. Se tratar� la herramienta que se utiliza actualmente y la que ha sido desarrollada para este tipo de proyectos.

\paragraph{Subversion}
Este sistema de control de versiones es un SVCC que se caracteriza por sus simplicidad, linealidad de versiones y gesti�n de su espacio como un �nico sistema de ficheros.

Como todo SCVC tiene una versi�n servidor, que se encarga principalmente de guardar las versiones en una base de datos en la que se pueden configurar permisos de accesos. Soporta varios protocolos de comunicaci�n con sus clientes, como son HTTP, HTTPS, SSH y SVN, su propio protocolo.

Suele tener problemas de configuraci�n de permisos de accesos, ya que todos los guardados, se hacen con el usuario del sistema que se est� utilizando, cambiando muchas veces los permisos, y haciendo que algunos desarrolladores cerraran el flujo a otros. Tambi�n es frecuente encontrar proyectos grandes con errores de integridad de la base de datos que resultan en una imposibilidad de recuperar determinadas versiones.

Algunos otros problemas son la necesidad de atomicidad del repositorio, que implica una indisponibilidad del repositorio durante su uso por otros desarrolladores, o la forma de manejar las colisiones de c�digo, que crean $3+n$ ficheros por colisi�n $n$.

\paragraph{Git}
Este es un SCVD se caracteriza por una total adaptaci�n al estilo de trabajo de los grupos de programadores. Como SCVD, tiene caracter�sticas de inerentes a ello, como son el hecho de que cada repositorio es independiente, posibilitando al desarrollador guardar su trabajo \textit{offline}. Tambi�n tiene facilidades para llevar un control de la sincronizaci�n con otros repositorios.

Como es distribuido, no hay una versi�n servidora y una cliente, son todos iguales. La base de datos es local, y est� integrada junto al resto de los datos del repositorio. El dise�o de Git es relativamente reciente, y se dise�o de manera que sirviera para el desarrollo de kernel, los mismos que desarrollan Linux.

Est� dise�ado con la integridad como factor principal. Todos las cosas se guardan en una base de datos referenciados por sus hashes, haciendo que cuando se requiere un objeto que ha sido guardado, se computa el hash durante la descompresi�n, verificando de esa manera que siempre se mantengan los archivos fielmente guardados.

El m�todo de desarrollo para el que est� dise�ado permite que un desarrollador pueda basar su trabajo en el trabajo que se hace en un repositorio oficial, y tener sus propias ramas de desarrollo con integraci�n continua.

\subsubsection{Criterio de la selecci�n}
\label{scv-criterios}
Para elegir correctamente el SCV que se va a utilizar, se tomar�n los siguientes criterios en cuenta.

\begin{description}
\item[{Uso y aprendizaje [10\%]}]
Uno de los principales problemas del desarrollo de un proyecto de kernel es todo el tiempo que se invierte en la familiarizaci�n con las herramientas de kernel. Al ser un proyecto tan peculiar, tiene muchas herramientas espec�ficamente creadas para ello.

La inversi�n de tiempo en la adaptaci�n a un nuevo sistema de control de versiones puede no ser la mejor opci�n, ya que puede no interesar dependiendo de la longitud del proyecto. En este caso, la inversi�n en tiempo puede merecer la pena.

\item[{Uso en proyectos similares [20\%]}]
Es importante ver las herramientas que utilizan el resto de proyectos similares, basados en kernel, para su desarrollo. Estos proyectos suelen tener en com�n la mayor parte de las herramientas utilizadas, al ser un �mbito de desarrollo tan espec�fico.

\item[{Flexibilidad [30\%]}]
Un elemento m�s importante que el uso y aprendizaje, que se pueden amortizar con el tiempo, es la posibilidad de que la citada herramienta provea de flexibilidad a la hora de trabajar. Uno de los principales hechos en el desarrollo de c�digo, es la posibilidad de hacer las cosas de diferentes maneras, y un sistema que facilite el desarrollo de las nuevas ideas es algo muy positivo para este tipo de proyectos.

Adem�s, el sistema tambi�n tiene que facilitar al m�ximos el desarrollo paralelo, y el concepto de Integraci�n Continua (\textit{CI: Continuous integration}), ya que como el proyecto oficial del kernel de Linux est� en continuo desarrollo y se modifican m�s de 500 l�neas de c�digo al d�a, es importante que el proyecto no quede desactualizado, para evitar tener que hacer otro proyecto que incluya unas modificaciones tan severas en el c�digo.

\item[{Control de cambios [40\%]}]
La caracter�stica m�s importante y a la que se le ha dado un mayor peso, es a la posibilidad de saber exactamente cuales son los cambios de una versi�n a otra, quien ha hecho determinados cambios, y sobre todo, la posibilidad de saber quien ha introducido qu� cambios y con qu� objetivo.

Hay muchas veces en las que los desarrolladores de kernel no documentan los cambios en las l�neas de c�digo a trav�s de comentarios, si no que comentan los cambios en los mensajes que se hacen al guardar. El acceso a la informaci�n que proporcionan los desarrolladores de esa manera es clave para la actualizaci�n del proyecto.

\end{description}

\subsubsection{Selecci�n de la soluci�n}
El sistema de control de versiones que mejor se ajusta a los requerimientos del proyecto es \textit{Git}, tal y como se puede observar en la siguiente tabla.
\begin{center}
\rowcolors{1}{gray}{white}
\begin{tabular}{|c|c|c|c|}
\hline Criterio & Ponderaci�n & Subversion & Git \\ 
\hiderowcolors
\hline Uso y aprendizaje & 10\% & 10/10 & 0/10 \\ 
\hline Uso en proyectos similares & 20\% & 1/20 & 20/20 \\ 
\hline Flexibilidad & 30\% & 10/30 & 30/30 \\ 
\hline Control de cambios & 40\% & 20/40 & 40/40 \\ 
\hline Total & 100\% & 51/100 & 90/100 \\ 
\hline 
\end{tabular}
\end{center}

En este proyecto se ha considerado que a�n cuando Subversion es m�s f�cil de utilizar, el hecho de que todos los proyectos centrados alrededor del desarrollo de kernel hayan utilizado Git hace que se pondere en positivo el esfuerzo requerido de aprendizaje.

Adem�s, las facilidades de git para el control de cambios, permiten ponerse en contacto f�cilmente con el desarrollador que ha participado en el c�digo.

\subsection{M�todo de trabajo y liberaci�n de c�digo}
En esta secci�n se va a estructurar, ya que es uno de los objetivos del proyecto, la manera en la que se van a guardar las modificaciones de c�digo y la forma en la que se va a organizar el trabajo del proyecto.

La manera de organizar esta secci�n ser� diferente a la seguida en otras ocasiones, ya que se organizar� como una explicaci�n de los m�todos de trabajo, y las formas de ponerlos en pr�ctica.

\subsubsection{Arquitectura del SCVD Git}
Actualmente el repositorio del proyecto se compone por un fichero de parche al kernel, que hay que aplicar y volver a crear cuando se ha seguido el desarrollo, para guardarlo dentro del SCV. Adem�s, en este repositorio se guarda el c�digo del m�dulo de kernel en el que est� contenido todo lo que no es estrictamente necesario de tener en el parche.

Aunque este m�todo de trabajo est� bien, las posibilidades que se abren al utilizar un sistema de control de versiones como Git es que se puede tener un rama de desarrollo dependiente de kernel, y otra paralela en el que se haga todo el desarrollo de una forma independiente, pero con posibilidad de ir integrando el trabajo.

El nuevo sistema, se puede permitir la integraci�n completa en el �rbol de c�digo de kernel, siendo de esta manera la m�s apropiada en caso de que se libere el c�digo.

\subsubsection{Integraci�n del c�digo en el �rbol de Linux}
Actualmente la manera de conseguir hacer funcionar la aplicaci�n ksensor en el �rbol de kernel es a trav�s de la aplicaci�n del parche, y posterior compilaci�n del kernel. Una vez parcheado, no hay una manera de no compilar esa parte.

Esta forma de trabajar complica la posibilidad de desarrollar utilidades complementarias pero independientes, pues no se puede discernir entre lo que pertenece al �rbol del kernel y lo que no.

Para facilitar el desarrollo, se optar� por integrar todo lo posible el trabajo llevado en el grupo de investigaci�n NQAS en el �rbol de kernel, para evitar tener que mantener la sincronizaci�n entre distintas versiones del c�digo. Por ello, se integrar� todo el c�digo en el repositorio.
% !TeX spellcheck = es_ES
% !TeX root = main.tex

\chapter{Dise�o}
Una vez planteados los requerimientos y especificaciones iniciales de la migraci�n de Ksensor, realizado el estudio de alternativas de los puntos m�s cr�ticos del dise�o y elegidas las alternativas �ptimas en cada caso, en este cap�tulo se describir� de forma breve el dise�o general de la soluci�n propuesta en el presente proyecto.

Tanto como no se han decidido cambios propiamente en el dise�o del sensor, habr�n de adaptarse la estructura del c�digo para su integraci�n propuesta, as� como migrar a una rama de desarrollo paralela al kernel de Linux.

No obstante, aunque no se haya cambiado el dise�o de alto nivel del proyecto anterior, se har�n cambios estructurales de bajo nivel. A continuaci�n, se describir�n los nuevos elementos que compondr�n el sistema.

\section{M�dulo de estad�sticas}
El m�dulo de estad�sticas, es uno diferente al mencionado en el proyecto ksensor \cite{KABO05}. Tanto como ese es de estad�sticas internas de Ksensor, este trata de sacar a la luz otras mediciones que son �tiles a la hora de diagnosticar la efectividad del sistema de captura. En este apartado se explicar�n primeramente las caracter�sticas principales de este m�dulo y despu�s se pasar� a explicar el funcionamiento general del mismo.

Durante su funcionamiento, llevar� la cuenta de varios datos que son interesantes para el diagn�stico del sistema, como la cantidad de veces que se han recogido paquetes, el n�mero de interrupciones de la tarjeta y otros datos que permiten hacer mediciones de como de efectiva es la captura.

Es importante remarcar que estos datos son los necesarios para hacer diagn�sticos de la efectividad del m�dulo de captura. Recoger estad�sticas que son propias del kernel pers� es algo que no se ha tenido en cuenta hasta ahora. Es algo necesario para poder medir la efectividad de la sonda.

Est� constituido por dos partes: las modificaciones en el kernel para hacer posible la creaci�n y medici�n de dichas estad�sticas; y el m�dulo que proporciona el acceso a las mismas.

El usuario del m�dulo, deber� activar una opci�n en la compilaci�n del kernel, y podr� insertarlo para observar las mediciones que se hagan a trav�s de unos ficheros virtuales espec�ficamente creados para tal fin.

\section{M�dulo de traceo}
Este m�dulo ha sido creado espec�ficamente para la monitorizaci�n de la distribuci�n de tiempos de captura. Ha sido creado porque tras haber analizado las estad�sticas que prove�a la anterior implementaci�n de Ksensor, se han visto defectos que hacen pensar que el sistema actual no es todo lo eficiente que puede ser.

Para ello, el m�dulo estar� tambi�n dividido en dos partes, como en el otro caso, la primera son las modificaciones que habr� que hacer para hacer las mediciones y guardar los tiempos de cada traza, y la segunda ser� el m�dulo que inicializar� la funci�n para salvar las trazas.

De esta manera, el usuario puede elegir cuando se emplea tiempo en guardar las trazas y cuando no. Adem�s, tendr� acotado en todo momento el tiempo en el que se hacen las trazas.

Este dise�o permite al usuario cargar y descargar el m�dulo, utilizando diferentes funciones que hagan cosas totalmente diferentes de las establecidas si se quiere, sin necesidad de recompilar el m�dulo.

A parte de la caracter�stica de dejar que el m�dulo haga lo que considere con los datos, el m�dulo dise�ado en un primer lugar, crea un fichero virtual para acceder a los datos en modo de lista, permitiendo de esa manera que el usuario pueda acceder a todos los datos en orden.
% !TeX spellcheck = es_ES
% !TeX root = main.tex

\chapter{Necesidades del sistema}
La complejidad del sistema que se quiere implementar exige que se elabore con detalle un conjunto de especificaciones de las necesidades del mismo, estando �stas reguladas por los objetivos que se plantearon en el cap�tulo \reference{cha:objetivos}{Objetivos}. Estas especificaciones servir�n de gran utilidad y ayuda como gu�a para el posterior dise�o del sistema, y asimismo, se utilizar�n para marcar cu�les son los l�mites del desarrollo del prototipo.

En este cap�tulo se desarrollar�n las especificaciones bajo dos puntos de vista. Por una parte, atendiendo a las necesidades de los posibles entornos de aplicaci�n de los resultados del proyecto, y por el otro, a las especificaciones funcionales de los elementos de la arquitectura. Primeramente, se describir� el sistema de forma general para que las especificaciones resulten m�s f�ciles de entender.

\section{Visi�n general del sistema}
Este proyecto tiene como principal objetivo el redise�o y reimplementaci�n de una sonda a nivel de kernel que permita capturar y analizar todos los paquetes que circulan por una red de alta velocidad. La captura de los paquetes ha de ser eficiente, y el an�lisis se ha de realizar de forma concurrente, si se dispone de una plataforma multiprocesador (SMP).

Por lo tanto, las especificaciones iniciales que se fijen tienen considerable importancia, ya que de su precisi�n y exigencia depender� en gran medida el �xito o fracaso del proyecto.

A partir del dise�o realizado se va a desarrollar un prototipo del sensor, el cual debe satisfacer las especificaciones funcionales que se indican en este cap�tulo. Mediante el prototipo se desea comprobar la funcionalidad del sistema y realizar una serie de pruebas orientadas a validar la hip�tesis de partida del proyecto, esto es, la mejora de prestaciones con respecto a Ksensor, a trav�s de la migraci�n de la sonda a una nueva versi�n de kernel y la aplicaci�n de correcciones para eliminar algunos defectos observados.

En el documento Pliego de Condiciones se establecer�n en detalle cu�les son las pruebas de validaci�n que el prototipo a desarrollar debe cumplir. El plan de pruebas se definir� antes de la implementaci�n y se emplear� para demostrar que el prototipo cumple el conjunto de especificaciones funcionales indicadas, que, l�gicamente, no tienen por qu� ser todas las que se imponen para el dise�o global del proyecto.

Por otro lado, una vez validado el prototipo, se proceder� a realizar una serie de pruebas de rendimiento en el sensor, con el objeto de poder evaluar la mejora de prestaciones que se obtiene en Ksensor gracias a la actualizaci�n. El plan de pruebas a seguir se definir� en el documento Estudio Comparativo de Rendimientos, en el cual se realizar� tambi�n un estudio comparativo a partir de los resultados obtenidos.

\section{Especificaciones de los entornos de aplicaci�n}
Los resultados de este proyecto deben estar enfocados para dar soluci�n a las necesidades que se plantean en el �mbito de la investigaci�n sobre la mejora de rendimientos en el an�lisis de tr�fico eficiente en redes de datos. En este sentido, los entornos de aplicaci�n que se plantean son los siguientes:

\begin{enumerate}
\item Entorno de desarrollo.
\item Entorno de pruebas.
\item Entorno de an�lisis comparativo.
\end{enumerate}

A continuaci�n se desarrollar�n las especificaciones particulares para cada uno de
estos entornos.

\subsection{Entorno de desarrollo}
El prototipo de Ksensor que se va a desarrollar en este proyecto se ejecutar� tambi�n en el nivel de kernel, con lo cual se deben considerar las particularidades, restricciones y complejidades que conlleva una implementaci�n de este tipo. En el kernel dif�cilmente pueden utilizarse herramientas de depuraci�n tipo gdb, y adem�s, un simple error o bug puede llegar a provocar el colapso total del sistema.

En consecuencia, ser� necesario reiniciar la m�quina, pero lo que es a�n peor, se destruir�n la mayor parte de los indicios o evidencias que pod�an apuntar el origen de dicho fallo. Es por ello que, en el desarrollo de Ksensor, se deben habilitar mecanismos de depuraci�n que permitan comprobar el funcionamiento de los m�dulos, adem�s de las facilidades de depuraci�n ya implementadas en el kernel, o bien en base a �stas.

A continuaci�n se describe el escenario en el que se desarrollar� el Ksensor. Se utilizar�n tres ordenadores, conectados entre s� a trav�s de una red independiente:
\begin{description}
\item[Gestor] Esta m�quina se utilizar� como estaci�n de trabajo, desde la cual se acceder� al resto de m�quinas de forma remota (v�a ssh).
\item[Sonda] Esta m�quina se utilizar� para la programaci�n, integraci�n y ejecuci�n de Ksensor. El sistema operativo ser� Debian GNU/Linux, en la versi�n del kernel 3.6.
\item[Inyector] Esta m�quina se utilizar� como inyector de tr�fico en las pruebas de validaci�n del prototipo, as� como en las pruebas de rendimiento.
\end{description}

La sonda y el inyector estar�n conectados entre s� a trav�s de un switch, en el que s�lo estar�n conectadas estas dos m�quinas, para poder controlar el tr�fico que se introduce en la red de captura. Ambas m�quinas se gestionar�n desde el gestor, utilizando una interfaz de red distinta. Se utilizar� esta configuraci�n para llevar a cabo el plan de pruebas de validaci�n definido en el Pliego de Condiciones.

La sonda se configurar� de tal manera que, cuando se produzca un error, se saque en los ficheros de registros \textit{logs} el mensaje correspondiente a trav�s del kernel. Por defecto, el mensaje se imprime en la pantalla y se vuelca a un fichero de \textit{logs}, gracias a la aplicaci�n syslogd.

No obstante, si el error es grave y el sistema colapsa (la m�quina deja de responder) los mensajes no se almacenar�n, con lo cual se pierde el motivo por el que se produjo el error. Para evitar esto, se podr�a utilizar el mecanismo netconsole implementado en Linux, mediante el cual se env�an los mensajes por medio de paquetes UDP, antes de que el sistema se cuelgue. La estaci�n de trabajo (gestor) ser�a el receptor o servidor de dichos mensajes, mientras que la sonda ser�a el cliente.

Otro de los mecanismos que puede ser implementado, aunque con algo m�s de tiempo dedicado, ser�a la posibilidad ejecutar un kernel de soporte para que en caso de fallo, tomara el control de la memoria, y guardara el estado completo de la memoria, para posteriormente ser enviado al desarrollador.

As� pues, las especificaciones a tener en cuenta en el desarrollo de la sonda son las siguientes:
\begin{itemize}
\item El kernel (Linux 3.6) se debe configurar con soporte para depuraci�n.
\item El prototipo de la sonda debe disponer de los mecanismos de depuraci�n necesarios para poder evaluar el funcionamiento del sistema con distinto nivel de detalle. En este sentido, se debe poder comprobar el funcionamiento global de la sonda, el de un m�dulo en concreto o el de una �nica funci�n.
\item El sistema debe ser capaz de enviar los mensajes de error que se produzcan a�n en el caso de que la m�quina se vaya a quedar bloqueada hasta que se ejecute un reinicio manual.
\item Se deben implementar funciones que permitan realizar pruebas unitarias sobre los m�dulos de la sonda, como por ejemplo, comprobar que se lee la l�gica de decisi�n de forma adecuada, que se desensamblan los paquetes correctamente, etc.
\item La sonda debe exportar diversas estad�sticas mientras se ejecuta, de modo que se pueda evaluar su comportamiento en determinadas circunstancias, por ejemplo, la forma en que se activa y desactiva el control de congesti�n.
\end{itemize}

\subsection{Entorno de pruebas}
El entorno de pruebas se refiere a las pruebas de rendimiento que se van a realizar en el laboratorio, con el objeto de valorar de forma cuantitativa la mejora de prestaciones que se obtiene con Ksensor en comparaci�n con la anterior implementaci�n del mismo. Para ello, se utilizar� la arquitectura para la automatizaci�n de pruebas dise�ada en \cite{AABS05}, adecu�ndola a las exigencias y caracter�sticas de nuestro sistema de captura y an�lisis de tr�fico de red.

El escenario en el que se realizar�n las pruebas ser� similar al descrito en la secci�n anterior. La sonda (Ksensor) y el inyector se conectar�n por medio de un switch, de tal forma que todo el tr�fico introducido en la red por el inyector ir� a parar a la sonda, el cual aplicar� la carga de an�lisis oportuna sobre los paquetes que captura.

Puesto que la finalidad principal de estas pruebas es obtener las caracter�sticas de rendimiento de las sondas, lo m�s interesante es comprobar el throughput de los mismos cuando se satura el enlace. Antes, se utilizaban varias m�quinas para saturar el enlace, ya que es muy dif�cil que un sistema operativo convencional pueda generar tal volumen de tr�fico.

Ahora, se dispone de inyectores hardware \cite{DAPR07} que tienen la capacidad de saturar un enlace a 1Gbps. Este tipo de inyectores no requieren otros equipos, por lo que se podr�a utilizar un enlace directo. En la arquitectura anterior se utilizaba un switch como forma de agregar tr�fico ethernet, pero se ha decidido no suprimir el switch para poder seguir disponiendo de unas estad�sticas externas.

As� pues, en esta configuraci�n se conectar�n dos m�quinas al switch: una para el sensor y una inyectora. En cierto modo, el inyector es el propio switch, por lo que en las pruebas de rendimiento ser� necesario obtener tambi�n las estad�sticas relativas a �ste, por llevar una contabilidad exacta por un medio externo.

Una vez planteado el escenario en el que se realizar�n las pruebas de rendimiento, se pueden definir ya las especificaciones que se deben tener en consideraci�n:
\begin{itemize}
\item La arquitectura de pruebas del sensor se basar� en la plataforma gen�rica
para la automatizaci�n de pruebas dise�ada en [BEAU05].
\item El sistema ha de ser capaz de lanzar bater�as de pruebas, configurando de
forma automatizada todos los agentes que participan en la prueba.
\item Se desarrollar� un agente para facilitar el encendido y apagado de ksensor
de forma automatizada.
\item Se proveer�n las utilidades necesarias para configurar los par�metros del
sensor y del inyector, as� como para obtener la informaci�n relativa a todos
los agentes que participan en la comunicaci�n.
\begin{itemize}
\item Par�metros de funcionamiento del sensor.
\begin{itemize}
\item Modos de captura.
\item Par�metros de la cola de paquetes.
\end{itemize}
\item Par�metros de funcionamiento del inyector
\begin{itemize}
\item Tasa de inyecci�n.
\end{itemize}
\item Estad�sticas de las interfaces de red.
\item Estad�sticas del switch.
\end{itemize}
\item Se deben dise�ar plantillas de pruebas para facilitar la realizaci�n de prue-
bas de rendimiento en laboratorio. Mediante estas plantillas se indicar�n los
par�metros con los que se van a ejecutar los agentes.
\end{itemize}

\subsection{Entorno de an�lisis comparativo}
A partir de los resultados obtenidos en las pruebas de rendimiento se puede realizar un estudio comparativo, con el fin de evaluar la mejora de prestaciones de ksensor. Para ello, se hace necesario definir primeramente un plan de las pruebas que se deben realizar, ya que de su precisi�n y exigencia depender� en gran medida el rigor de las conclusiones que se vayan a obtener. El plan de pruebas se detalla en la secci�n Plan de Pruebas en el documento Estudio Comparativo de Rendimientos, pero en este apartado se aportar�n una serie de especificaciones iniciales de las mismas.

Las pruebas de rendimiento que se realicen deben permitir cuantificar el throughput (tasa de paquetes procesados) al ser sometidos a diferentes tasas de inyecci�n, especialmente para cargas de tr�fico elevadas, con el objeto de poder conocer su comportamiento en redes de alta velocidad. Las pruebas deben facilitar, asimismo, conocer los par�metros de configuraci�n �ptimos en las diferentes versiones de ksensor. A partir de los resultados que se obtengan, se podr�n realizar gr�ficas comparativas, utilizando las herramientas necesarias para dicha tarea.

Por tanto, las especificaciones que se han de tener en cuenta al realizar el an�lisis comparativo entre las diferentes versiones son:
\begin{itemize}
\item Se deben identificar los par�metros susceptibles de estudio.
\item Se deben definir los prototipos de ensayo, a fin de realizar:
\begin{itemize}
\item Comparativa de las diferentes versiones de ksensor.
\item Comparativa de las diferencias entre los controles de congesti�n.
\item Comparativa de rendimiento para diferente n�mero de procesadores.
\end{itemize}
\item Juegos de ensayo: Para poder realizar un an�lisis comparativo de los diferentes prototipos y de las diferentes configuraciones, es necesario definir, en cada caso, la colecci�n de pruebas a realizar en el laboratorio.
\item Se deben proveer las utilidades necesarias para poder dibujar gr�ficas comparativas a partir de los resultados obtenidos en laboratorio.
\end{itemize}

\subsection{Especificaciones funcionales del sistema}
En esta secci�n se pretende ofrecer una descripci�n de las especificaciones o requerimientos que deben fijarse para el dise�o del sistema, de modo que el cumplimiento de �stas conduzca a la consecuci�n de los objetivos planteados al comienzo del presente documento.

Los m�dulos que se relatan tienen val�a por si mismos, ya que no son de uso exclusivo junto con Ksensor, y por lo tanto, los requerimientos ir�n orientados hacia un uso m�s general que Ksensor. A continuaci�n, se desglosar�n las caracter�sticas m�s importantes de todos los m�dulos nuevos y se mencionar�n las especificaciones de cada uno de ellos:
\begin{itemize}
\item M�dulo estad�sticas.
\item M�dulo traceo.
\end{itemize}

\subsection{M�dulo de estad�sticas}
El m�dulo de estad�sticas de que se va a implementar debe hacer mediciones sobre los par�metros que se considerar�n a continuaci�n. Tambi�n deber� proveer una interfaz sencilla de acceso a los m�smos, permitiendo la recogida de datos despu�s de cada prueba.

Los datos relevantes a ser estudiados deber�an ser:
\begin{itemize}
\item Los tiempos totales empleados en:
\begin{itemize}
\item Rutinas de servicio a la interrupci�n de captura de red.
\item Captura de paquetes por interfaz.
\item Capturar cada paquete.
\end{itemize}
\item N�meros absolutos de:
\begin{itemize}
\item Rutinas de servicio a la interrupci�n de captura de red.
\item Capturas de paquetes por cada interfaz.
\item Paquetes procesados para su captura.
\end{itemize}
\end{itemize}


\subsection{M�dulo de traceo}
La finalidad del m�dulo de traceo es, como su propio nombre indica, llevar a cabo trazas. Debe ser hecho de una manera simple, extensible y modular. La idea detr�s de este m�dulo es que se pueda ampliar para posibles necesidades de trazas que puedan surgir. Incluso, si en alg�n momento fuera necesario, a�adir varias trazas diferentes utilizando el mismo sistema.

La existencia de este m�dulo se propone como manera de comprobar modelos te�ricos de como los paquetes son recibidos en el sistema, pero pueden ser utilizados con muchos otros fines.

Este m�dulo deber� cumplir las siguientes premisas:
\begin{itemize}
\item Deber� ser lo m�s simple posible.
\item Tiene que ser r�pido y eficiente, sin gastar mucho tiempo, ya que en rutinas prioritarias podr�a tener un impacto grave.
\item Tener una interfaz f�cil de acceder sin necesidad de crear un mapa de memoria personalizado.
\end{itemize}
% !TeX spellcheck = es_ES
% !TeX root = main.tex

\chapter{Gesti�n}
En este cap�tulo se resumen las tareas de gesti�n que se deben desempe�ar en la realizaci�n del presente proyecto. Se expondr� el resumen del plan de trabajo que se debe seguir y el resumen del presupuesto. En primer lugar, se presentar� el grupo humano de trabajo que participar� en la realizaci�n de los mismos, el cual es com�n en ambos casos.

\section{Equipo de trabajo}
Para el desarrollo del presente proyecto y el del proyecto de continuaci�n se debe contar con el siguiente equipo de trabajo:
\begin{center}
\rowcolors{1}{gray}{white}
\begin{tabular}{|c|c|p{5cm}|}
\hline C�DIGO & RESPONSABILIDAD & DESCRIPCI�N \\ 
\hiderowcolors
\hline H1 & Director de proyecto & Ser� el encargado de supervisar los dise�os y de que se cumpla con el plan de trabajo. \\ 
\hline H2 & Dise�ador & Ser� el encargado de realizar el dise�o de los m�dulos del proyecto y los planes de pruebas de rendimiento. \\ 
\hline H3 & Programador & Se encargar� de implementar los m�dulos dise�ados, y de llevar a cabo las pruebas de validaci�n y de rendimiento previstas. \\ 
\hline H4 & Administrativo & Se encargar� de la documentaci�n del proyecto, y de resolver las cuestiones burocr�ticas y administrativas.\\ 
\hline
\end{tabular}
\end{center}

\section{Resumen del proyecto}
En esta secci�n se presenta el resumen del Plan de Trabajo y Presupuesto para la elaboraci�n del proyecto Dise�o e implementaci�n de un sensor a nivel de kernel para el an�lisis de tr�fico en redes de alta velocidad.

\subsection{Plan de trabajo}
En este apartado se definen los paquetes de trabajo que se van a seguir en la realizaci�n del proyecto. Cada paquete de trabajo se desglosa en las tareas que se han de realizar para su consecuci�n. Se indicar�n, asimismo, las unidades de entrega correspondientes a cada paquete de trabajo o tarea.

\begin{description}
\item [PT1] Estudio de las necesidades del sistema.
\begin{description}
\item [UE1] Definici�n de las especificaciones
\begin{description}
\item [T.101] Estudio de proyectos relacionados.
\item [T.102] Definici�n de las especificaciones.
\end{description}
\end{description}
\item [PT2] Estudio de alternativas.
\begin{description}
\item [UE2] Estudio de las alternativas de dise�o
\begin{description}
\item [T.201] Estudio de mecanismos de comunicaci�n con el espacio de usuario.
\item [T.202] Estudio de las metodolog�as de programaci�n en el kernel.
\end{description}
\end{description}
\item [PT3] Dise�o de los m�dulos del proyecto.
\begin{description}
\item [UE3] Redise�o de los componentes del sistema
\begin{description}
\item [T.301] Redise�o del m�dulo de captura.
\item [T.302] Redise�o del m�dulo de procesamiento.
\item [T.303] Redise�o del driver.
\item [T.304] Dise�o del m�dulo de estad�sticas.
\item [T.305] Dise�o del m�dulo de traceo.
\item [T.306] Dise�o de la jerarqu�a de integraci�n.
\end{description}
\end{description}
\item [PT4] Desarrollo de los m�dulos del proyecto.
\begin{description}
\item [UE4] Desarrollo y depuraci�n de los m�dulos del proyecto
\begin{description}
\item [T.401] Programaci�n y pruebas del m�dulo de captura.
\item [T.402] Programaci�n y pruebas del m�dulo de procesamiento.
\item [T.403] Programaci�n y pruebas del driver.
\item [T.404] Programaci�n y pruebas del m�dulo de estad�sticas.
\item [T.405] Programaci�n y pruebas del m�dulo de traceo.
\item [T.406] Programaci�n y pruebas de la jerarqu�a de integraci�n.
\end{description}
\end{description}
\item [PT5] Integraci�n de los m�dulos, pruebas de validaci�n y de rendimiento.
\begin{description}
\item [T.501] Integraci�n de los m�dulos del proyecto.
\begin{description}
\item [UE501] Integraci�n de los componentes de la sonda.
\end{description}
\item [T.502] Pruebas de validaci�n y estabilidad del sistema.
\begin{description}
\item [UE502] Pruebas de validaci�n del prototipo.
\end{description}
\item [T.503] Pruebas comparativas de rendimiento.
\begin{description}
\item [UE503] Pruebas comparativas de rendimiento
\end{description}
\end{description}
\item [PT6] Gesti�n del proyecto.
\begin{description}
\item [UE6] Documentaci�n final del proyecto.
\begin{description}
\item [T.601] Tareas de gesti�n.
\item [T.602] Documentaci�n.
\end{description}
\end{description}
\end{description}


\subsection{Presupuesto}
En este apartado, se mostrar� el resumen del presupuesto correspondiente al plan de trabajo de la secci�n anterior. Un presupuesto m�s detallado se puede encontrar en el documento anexo Plan de Trabajo y Presupuesto del Estudio de Viabilidad.

El presupuesto est� desglosado en presupuestos parciales de recursos humanos y recursos materiales. En la siguiente tabla se condensa toda la informaci�n relativa al presupuesto. Se le a�ade un 2\% para posibles imprevistos que puedan surgir a lo largo de la ejecuci�n, y al ser un proyecto interno no se le a�ade el IVA:

\begin{center}
\rowcolors{2}{gray}{white}
\begin{tabular}{|c|c|}
\hline CONCEPTO & COSTE TOTAL\\
\hiderowcolors
\hline Total recursos humanos & 50.240,00 \euro \\
\hline Total recursos materiales & 1.402,01 \euro \\
\hline Total parcial & 51.642,01 \euro \\
\hline Imprevistos (2\%) & 1.032,84 \euro \\
\hline Presupuesto Total & 52.674,85 \euro \\
\hline
\end{tabular}
\end{center}
El coste para el proyecto \textit{Adaptaci�n y extensi�n de la herramienta de captura y an�lisis de tr�fico Ksensor} asciende a una cantidad total de cincuenta y dos mil seiscientos setenta y cuatro euros y ochenta y cinco c�ntimos.

% !TeX spellcheck = es_ES
% !TeX root = main.tex

\section{Riesgos}


\pagenumbering{Roman}

\bibliography{bibliography}

\printindex

\end{document}