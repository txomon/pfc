% !TeX spellcheck = es_ES
\documentclass[a4paper]{report} % Para tener un documento grande pero sin ser un libro (solo hay esta posibilidad)
\usepackage[latin1]{inputenc}   % Para la codificaci�n
\usepackage{textcomp}           % Para los simbolos de euro
\usepackage[spanish]{babel}     % Para el lenguaje que tenga bien
\usepackage[T1]{fontenc}        %      los saltos de linea con gui�n etc.
\bibliographystyle{alpha}       % Estilo de bibliograf�a
\usepackage{makeidx}            % Hacer �ndices
\usepackage{fancyhdr}           % Cambiar las cabeceras del documento
\usepackage{graphicx}           % Para insertar im�genes
\graphicspath{{./images/}}      % La ruta relativa de las im�genes
\usepackage{hyperref}           % Hacer links guays entre secciones y hacia fuera
\usepackage[hypcap]{caption}    % Hacer que las referencias se vean, y no se queden por encima (referenciando imagenes, etc.)
\hypersetup{
	colorlinks=true,
	linkcolor=blue,
	citecolor=red,
}

% % % % % % % % % % % % % % % % % % % % % % %
% Secci�n de comandos personalizados        %
% % % % % % % % % % % % % % % % % % % % % % %
\newcommand{\reference}[2]{\hyperref[#1]{\textit{\ref*{#1} #2}}}

\title{Adaptaci�n y extensi�n de la herramienta de captura y an�lisis de tr�fico KSENSOR}
\author{Javier Domingo Cansino}
\makeindex

\begin{document}
\pagenumbering{Roman}
\maketitle % Crear el t�tulo
\tableofcontents % Crear la tabla de contenidos

\pagenumbering{arabic}
% !TeX spellcheck = es_ES
% !TeX root = main.tex
En los últimos años, las tecnologías informáticas de la comunicación están revolucionando la manera de comunicarse del mundo entero, tener acceso a internet se ha convertido en algo necesario para poder ponerse en contacto con el resto del mundo.

Cada vez está todo más interconectado, con el consiguiente crecimiento del tráfico en todas las redes. Incluso empresas que antes contrataban redes privadas para sus comunicaciones, ahora utilizan internet.

Como en cualquier servicio crítico el mantenimiento debe ser proactivo y por ello se han creado métodos para garantizar la máxima eficiencia de las comunicaciones. Para ello se debe garantizar la seguridad en la red y la detección instantánea de problemas, asegurando así la \textit{QoS} (\textit{Quality of Service}, Calidad de Servicio).

Para la securización de la red se implantan \textit{firewalls}, \textit{proxys}, y \textit{IDS}s (\textit{Intrusion Detection System}, Sistema de Detección de Intrusión), que se encargan de evitar accesos externos no autorizados a la red, controlar el tráfico saliente, y detectar posibles atacantes dentro de la red.

El método para asegurar la calidad de las comunicaciones es la medición de parámetros de \textit{QoS}, tales como retardo de los paquetes, errores en la transmisión, tipos de tráfico comunes, rutas óptimas, etc.

Son varias las soluciones comerciales que implementan estos servicios pero actualmente no hay ninguna que implemente todo en un mismo sistema. Además, no hay ningún producto de código libre que aproveche al máximo la posible eficiencia del equipo, haciendo un análisis \textit{online}\footnote{Online se refiere a hacer el análisis mientras se captura, en vez de capturar, guardar y luego analizar.}.

En el grupo de investigación \textit{Network Quality and Security (NQAS)} se ha implementado un sensor que implementa algunas de las citadas funcionalidades, en el diseño se ha contemplado la posibilidad de más adelante poder ampliarlo de una manera sencilla, y como valor añadido fundamental, aprovecha al máximo los recursos disponibles, ya que ha sido programado de una manera que permita ejecución multi-hilo.

Este prototipo se llama ksensor, aunque no es el primero que se ha creado. La línea de investigación principal, llamada hi-sensor, ha ido haciendo cada vez más complejo el diseño y la programación del prototipo con el objetivo de mejorar la eficacia.

A rasgos generales, primero se diseño de forma que la aplicación fuera un programa de usuario que capturaba a través de las llamadas estándar al sistema. Se utilizaron también las librerías para tener computación multi-hilo, y se consiguieron unos resultados prometedores.

Cuando se empezaron a analizar redes de alta velocidad en saturación, y se vio que había un problema: El equipo estaba continuamente capturando, y no había espacio para el análisis. Para solucionar este problema, se acordó migrar a espacio de kernel.

El kernel es la base que hace que todos los programas funcionen. Es un programa que se encarga de gestionar los dispositivos y recursos del ordenador para ofrecer una interfaz libre de detalles a los programas. Además, también se encarga de hacer que varios programas se ejecuten como si se ejecutaran a la vez, que los programas sean capaces de direccionar una memoria que no existe de una forma transparente a ellos, manejando eficazmente una MMU (\textit{Memory Management Unit}, Unidad de Gestión de Memoria), y muchas otras cosas ocultas al programador más.

En general, del diseño de un buen kernel dependerá la eficiencia del sistema. En nuestro caso, esta característica tiene valor añadido porque el diseño del kernel no está hecho para nuestro tipo de sistema. La eficiencia del kernel es buena para los sistemas operativos de carácter general, en los que capturar los paquetes de uno mismo es suficiente, y que es una cantidad pequeña en comparación con el tráfico de la red.

El sistema que tenemos es únicamente para capturar y procesar y por lo tanto, la máxima eficiencia se puede describir como el equilibrio entre el tiempo en el que el ordenador esta capturando y el que está analizando. Se tienen que capturar exactamente el número de paquetes que se van a analizar.

Ahora, el sistema es una
% !TeX spellcheck = es_ES
% !TeX root = main.tex

\chapter{Objetivos}
\label{cha:objetivos}
El objetivo principal del proyecto es la \textbf{adaptaci�n del programa Ksensor y modificaciones asociadas en kernel a una versi�n actual y el desarrollo de herramientas para la validaci�n del prototipo}. El dise�o actual es el descrito en el proyecto KSENSOR\cite{KABO05}.

Por lo tanto, los objetivos a conseguir mediante este proyecto son:

\begin{enumerate}
\item Captura eficiente del tr�fico de red
\item Paralelizaci�n del an�lisis
\item Compatibilidad con los m�dulos externos a la sonda
\item Estudio comparativo de rendimientos
\end{enumerate}

\section{Captura eficiente del tr�fico de red}
Ksensor debe optimizar el proceso de captura de los paquetes, eliminando si es preciso acciones realizadas por el kernel que no son estrictamente necesarias para el buen funcionamiento del sistema (recolecci�n de estad�sticas, copia del paquete a espacio de usuario, etc.).

El kernel de Linux es de prop�sito general, por lo que no est� espec�ficamente dise�ado para realizar una captura eficiente del tr�fico. Ya que la mayor parte de los sistemas de an�lisis de tr�fico a nivel de usuario se apoyan para realizar la captura en las facilidades proporcionadas por sistemas operativos de prop�sito general, su rendimiento no es �ptimo. La sonda que se propone en este proyecto debe, al trabajar directamente desde el nivel de kernel, resolver dichas ineficiencias, evitando por ejemplo capturar paquetes que posteriormente no pueda analizar, lo cual sin duda degrada el rendimiento.

\section{Paralelizaci�n del an�lisis}
El an�lisis del tr�fico en redes de alta velocidad es un proceso costoso, que le exige a la sonda un alto rendimiento y una capacidad de procesamiento que no se puede conseguir con las arquitecturas convencionales. Con el fin de acomodar ksensor a las necesidades de procesamiento requeridas por la red, en este proyecto se propone mejorar las prestaciones de una m�quina aumentando el n�mero de procesadores que ejecutan an�lisis de red.

As� pues, con un dise�o en base a plataformas multiprocesador, se podr� disponer de m�ltiples instancias de an�lisis en la sonda, trabajando de forma concurrente sobre el tr�fico capturado. El sistema dise�ado deber� permitir, en la medida de lo posible, el paralelismo en el procesado de los paquetes, a fin de obtener un alto rendimiento. Se deber� resolver tambi�n, la sincronizaci�n en el acceso a la informaci�n compartida entre instancias.

\section{Compatibilidad con los m�dulos externos la sonda}
Debemos tener en cuenta que la sonda es tan s�lo el n�cleo del sistema de an�lisis de tr�fico, y que para su buen funcionamiento necesita trabajar de forma coordinada con el resto de componentes del sistema, en particular, con el parser y el m�dulo de procesamiento offline\cite{AIDG03}, los cuales necesitan acceder al mapa de memoria de Ksensor para realizar su cometido. Por esta raz�n, la sonda debe ofrecer una interfaz unificada y gen�rica para el acceso a su mapa de memoria desde el espacio de usuario y controlar el acceso al mismo a trav�s de alg�n mecanismo de sincronizaci�n. El dise�o que se realice deber� contemplar adem�s la posibilidad de que existan nuevos m�dulos en el futuro.

\section{Estudio comparativo de rendimientos}
La sonda a nivel de kernel (Ksensor) pretende resolver las limitaciones inherentes al dise�o del sensor en espacio de usuario (Adviser), con el fin de obtener una mejora de prestaciones que posibiliten el an�lisis del tr�fico en redes de alta velocidad. Sin embargo, una implementaci�n deficiente o, quiz�s, la existencia de factores que no se han considerado podr�an impedir obtener los resultados esperados. Por lo tanto, se hace necesario poder comprobar y cuantificar las mejoras de rendimiento obtenidas, para poder valorar la viabilidad de un futuro producto comercial.

En este proyecto se reimplementar� un prototipo de Ksensor, el cual permitir� realizar las pruebas de rendimiento pertinentes, con el fin de poder valorar los resultados y comprobar las mejoras obtenidas, si las hubiera. Las pruebas de rendimiento permitir�n, asimismo, conocer los par�metros de configuraci�n �ptimos en la sonda.

Se debe tener en cuenta, que el rendimiento debe ser al menos, mejor que la anterior implementaci�n, ya que en este caso se tiene un punto de partida. Para la comparaci�n, se implementar�n herramientas de medici�n del patr�n de tiempos para poder diagnosticar efectos que no se hayan podido observar en el dise�o.
% !TeX spellcheck = es_ES
% !TeX root = main.tex

\chapter{Beneficios}
En esta secci�n se presentan los principales beneficios que aporta el proyecto, los cuales se engloban fundamentalmente en tres campos:
\begin{enumerate}
\item Beneficios t�cnicos
\item Beneficios econ�micos
\item Beneficios sociales
\end{enumerate}

\section{Beneficios t�cnicos}
La ejecuci�n del proyecto planteado aporta una serie de beneficios t�cnicos en el campo del an�lisis de tr�fico de red. A continuaci�n, se describe individualmente cada uno de ellos.

\subsection{An�lisis del tr�fico en redes de alta velocidad}
Optimizando el proceso de captura de los paquetes, el sensor a nivel de kernel permitir� realizar el an�lisis del tr�fico en redes de alta velocidad. �ste es el principal objetivo que se ha definido en el proyecto, as� como el principal beneficio t�cnico que se espera obtener. 
El an�lisis del tr�fico en redes de alta velocidad exige un coste computacional elevado, que no es posible obtener mediante las arquitecturas cl�sicas. Dicho problema se encuentra actualmente en fase de investigaci�n y es objeto de estudio por la comunidad cient�fica. 
El proyecto que se plantea propone una soluci�n software que, al funcionar desde el nivel de kernel, permitir� efectuar una captura eficiente de los paquetes, y al trabajar sobre plataformas multiprocesador, realizar el an�lisis concurrente de los mismos.

\subsection{Integraci�n de tipos de an�lisis}
Ksensor ha sido dise�ado de forma que pueda realizar cualquier tipo de an�lisis sobre el tr�fico capturado. Cuando se captura un paquete de la red, la sonda lo procesa atendiendo a la l�gica de procesamiento cargada en memoria por el parser. Dicho mapa de memoria se organiza en la forma de un �rbol de nodos de decisi�n, que el motor de an�lisis del sensor recorre para determinar el tratamiento que se le ha de dar al paquete.

El dise�o de la l�gica de nodos es gen�rico y multidisciplinar, por lo que permite efectuar cualquier tipo de tratamiento sobre el tr�fico capturado. As� pues, el sensor es v�lido en diferentes �mbitos de la telem�tica como la detecci�n de intrusi�n, la monitorizaci�n inteligente de red o la medida de par�metros de calidad de servicio.

\subsection{Arquitectura de pruebas}
Con el fin de poder realizar pruebas de rendimiento en el sensor de forma automatizada, en este proyecto se ha desarrollado una arquitectura de pruebas, basada en la plataforma gen�rica dise�ada en \cite{AABS05}. Esta arquitectura permite comprobar el comportamiento tanto de Adviser como de Ksensor cuando se los somete a un patr�n de tr�fico determinado. Para ello se utilizan plantillas de pruebas o tests, que el sistema ejecuta de forma automatizada y sucesiva, configurando la sonda y los inyectores seg�n se especifique en la plantilla.

En la arquitectura actual, se dispone de inyectores hardware, llamados DAG, que son capaces de inyectar a una tasa capaz de saturar el enlace, permitiendo probar el sistema bajo condiciones de estr�s, que se dan en redes de alta velocidad. Asimismo, la arquitectura recoge diversas estad�sticas de las pruebas realizadas, de cada uno de los componentes que participan en las mismas: switch, interfaces de red, inyectores, sonda, etc.


\section{Beneficios econ�micos}
La utilizaci�n del sistema dise�ado como herramienta de an�lisis de tr�fico introduce un conjunto de beneficios econ�micos en relaci�n con las aportaciones t�cnicas que se han expuesto anteriormente.

\subsection{An�lisis del tr�fico en redes de alta velocidad}
En la actualidad no existen apenas herramientas en el mercado capaces de realizar el an�lisis del tr�fico en redes de alta velocidad. Ksensor ha sido espec�ficamente dise�ado para satisfacer las necesidades de captura y procesamiento t�picas en estas redes. As�, un producto derivado del prototipo de Ksensor podr�a ser utilizado en los segmentos de alta velocidad de los Proveedores de Servicios de Internet (ISP), en los cuales podr�a resultar �til la medida de par�metros de QoS o la monitorizaci�n del tr�fico cursante. Podr�a ser utilizado, tambi�n, como herramienta de an�lisis en las redes de �rea local de las empresas.

\subsubsection{Localizaci�n de fallos de red}
El buen funcionamiento de las redes corporativas es indispensable en el ejercicio empresarial. Esto es especialmente cierto en la actualidad, en que la mayor parte de las transacciones se realizan de forma telem�tica. Si por alg�n motivo la red completa cae puede significar un par�n en el proceso productivo. La indisponibilidad del servicio impide que los clientes puedan contactar con la empresa o que se confirmen los pedidos. En cualquier caso, afectar� negativamente a la producci�n y, en periodos de indisponibilidad largos, degradar� su imagen.

Haciendo uso de la sonda se podr�a reducir el tiempo entre que ocurre el fallo y se arregla. La sonda analizar�a todo el tr�fico que circula por la red y podr�a programarse por ejemplo para monitorizar las conexiones con el servidor web. Si �ste fallara, la sonda lo detectar�a y podr�a generar una alarma para indicarlo. De este modo, el administrador de la red podr�a utilizar la sonda como herramienta de diagn�stico para localizar el origen del problema.

En un caso de uso m�s avanzado, se podr�a incluso llegar a detectar los fallos, sin llegar a caer el servicio. Por ejemplo, se podr�a programar la sonda para que analizara las retransmisiones, el retardo y par�metros de rutado de paquetes, resultando en que en caso de detectarse un n�mero de paquetes con el TTL a punto de expirar, demasiadas retransmisiones o cualquier otro s�ntoma de inestabilidad en la red, avisara al administrador, o incluso, activara directivas de balanceo de tr�fico.

\subsubsection{Detecci�n de ataques en la red}
La seguridad es uno de los factores clave en el dise�o de las redes corporativas. De no preservarse la integridad de las comunicaciones, personas ajenas a la empresa podr�an obtener informaci�n privada o amenazar de alg�n modo su funcionamiento normal. Si bien ya existen dispositivos espec�ficos con este fin, como los firewall, debe tenerse en cuenta que gran parte de los ataques proceden del interior de la empresa, y que por tanto pueden burlar sin problemas las medidas de seguridad perimetral. La sonda, mediante el an�lisis del tr�fico de la red, podr�a realizar la labor de un sistema de detecci�n de intrusi�n, buscando patrones de tr�fico que considera sospechosos, independientemente de su procedencia, y por tanto �til tambi�n en estos casos.

\subsubsection{Reducci�n de los dispositivos necesarios}
El dise�o de Ksensor permite integrar en un �nico dispositivo diferentes tipos de an�lisis, que de lo contrario �nicamente podr�an realizarse utilizando un conjunto de
sistemas espec�ficos. As�, la sonda puede obtener diversos par�metros de la red, monitorizar las conexiones o incluso buscar patrones de tr�fico al igual que en los sistemas de detecci�n de intrusi�n.


\section{Beneficios sociales}

Los beneficios sociales que aporta la sonda se derivan principalmente de la utilizaci�n de la arquitectura automatizada de pruebas, expuesta anteriormente en el apartado de beneficios t�cnicos.

\subsection{B�squeda de par�metros �ptimos en la sonda}
La arquitectura de pruebas puede utilizarse para comprobar el funcionamiento de la sonda con diferentes par�metros de configuraci�n, sometido a diversas cargas de red que tambi�n es posible especificar. De este modo, se pueden obtener de forma emp�rica los par�metros �ptimos configurables en el sensor, como puede ser, por ejemplo, el tama�o de la cola en la que se almacenan los paquetes capturados, el n�mero �ptimo de tarjetas de red, o la frecuencia de captura de paquetes.

\subsection{Comparativa de rendimientos entre sondas}
La arquitectura de pruebas se puede utilizar para realizar una comparativa de rendimiento entre las sondas a nivel de kernel ksensor y su nueva implementaci�n actualizada, con la finalidad de apreciar el comportamiento que �stos presentan cuando se los somete a un patr�n de tr�fico determinado. Tal y como la ejecuci�n de dichas pruebas permiti�, asimismo, justificar la realizaci�n de este proyecto de continuaci�n, al haberse obtenido resultados favorables en el rendimiento de Ksensor.

\subsection{Reducci�n del tiempo invertido en pruebas}
Las pruebas se realizan de forma automatizada, con lo cual �nicamente es necesario definir una bater�a de pruebas y el sistema las ejecutar� sucesivamente en el orden que se hayan especificado. En consecuencia, el tiempo que una persona ha de invertir en realizar las pruebas se reduce notablemente y por lo tanto reduce los costes del proyecto.

% !TeX spellcheck = es_ES
% !TeX root = main.tex

\chapter{An�lisis de alternativas}
En este cap�tulo se expondr�n primeramente las especificaciones y requerimientos b�sicos del presente proyecto, describiendo brevemente la arquitectura inicial de la cual se parte en el dise�o de la sonda e indicando los requerimientos que se le imponen inicialmente.

Se expondr�n asimismo las mejoras que se pretenden conseguir con el nuevo dise�o, indicando los requisitos que debe cumplir el kernel del sistema operativo para llevarlas a cabo. A continuaci�n, se expondr�n las alternativas de dise�o que se deben considerar en la realizaci�n del proyecto y se har� un estudio para determinar cu�l es la �ptima en cada caso.

A lo largo del dise�o se han de tomar decisiones que van a contribuir de forma decisiva al �xito o fracaso del proyecto. Por esta raz�n, conviene analizar minuciosamente las alternativas que existen. No es necesario tener en cuenta todas y cada una de las posibilidades pero s� las que puedan resultar cr�ticas a la hora de aportar una soluci�n.

As� pues, se partir� de los requerimientos apuntados en el primer apartado para a continuaci�n plantear las alternativas concernientes a los puntos m�s cr�ticos del dise�o, y una vez tomadas las decisiones oportunas, se pasar�, en el cap�tulo de Dise�o, a describir con mayor detenimiento la soluci�n general propuesta, incidiendo especialmente en las partes previamente discutidas.

\section{Especificaciones y requerimientos b�sicos}
En el presente proyecto se plantea el desarrollo de una sonda de an�lisis de tr�fico a nivel de kernel, para lo cual se parte de un dise�o previo de la sonda (Ksensor), realizado en un proyecto anterior \cite{KABO05}. La sonda es el n�cleo de una arquitectura dedicada al an�lisis de tr�fico, la cual incluye adem�s de la sonda, algunos m�dulos externos a �ste pero que tienen relaci�n directa con �l, como el parser o el OPM. As� pues, se hace necesario considerar los requerimientos que imponen estos m�dulos, ya que pueden condicionar en buena parte el dise�o que se realice en el nivel de kernel.

Por otro lado, conviene especificar las mejoras de dise�o que se pretenden obtener con la migraci�n de Ksensor, en base a las cuales se ha justificado la realizaci�n de este proyecto. El cumplimiento o no de dichas mejoras condicionar�n el �xito o fracaso del proyecto, por lo que se plantear�n tambi�n las mejoras que ha tenido el kernel de Linux para poder obtenerlas.

\subsection{Arquitectura de partida}

\begin{figure}
\centering
\def\svgwidth{\columnwidth}
\includesvg{images/arquitectura-previa}
\end{figure}

% !TeX spellcheck = es_ES
% !TeX root = main.tex

\section{Dise\~{n}o}

% !TeX spellcheck = es_ES
% !TeX root = main.tex

\chapter{Necesidades del sistema}
La complejidad del sistema que se quiere implementar exige que se elabore con detalle un conjunto de especificaciones de las necesidades del mismo, estando �stas reguladas por los objetivos que se plantearon en el cap�tulo \reference{cha:objetivos}{Objetivos}. Estas especificaciones servir�n de gran utilidad y ayuda como gu�a para el posterior dise�o del sistema, y asimismo, se utilizar�n para marcar cu�les son los l�mites del desarrollo del prototipo.

En este cap�tulo se desarrollar�n las especificaciones bajo dos puntos de vista. Por una parte, atendiendo a las necesidades de los posibles entornos de aplicaci�n de los resultados del proyecto, y por el otro, a las especificaciones funcionales de los elementos de la arquitectura. Primeramente, se describir� el sistema de forma general para que las especificaciones resulten m�s f�ciles de entender.

\section{Visi�n general del sistema}
Este proyecto tiene como principal objetivo el redise�o y reimplementaci�n de una sonda a nivel de kernel que permita capturar y analizar todos los paquetes que circulan por una red de alta velocidad. La captura de los paquetes ha de ser eficiente, y el an�lisis se ha de realizar de forma concurrente, si se dispone de una plataforma multiprocesador (SMP).

Por lo tanto, las especificaciones iniciales que se fijen tienen considerable importancia, ya que de su precisi�n y exigencia depender� en gran medida el �xito o fracaso del proyecto.

A partir del dise�o realizado se va a desarrollar un prototipo del sensor, el cual debe satisfacer las especificaciones funcionales que se indican en este cap�tulo. Mediante el prototipo se desea comprobar la funcionalidad del sistema y realizar una serie de pruebas orientadas a validar la hip�tesis de partida del proyecto, esto es, la mejora de prestaciones con respecto a Ksensor, a trav�s de la migraci�n de la sonda a una nueva versi�n de kernel y la aplicaci�n de correcciones para eliminar algunos defectos observados.

En el documento Pliego de Condiciones se establecer�n en detalle cu�les son las pruebas de validaci�n que el prototipo a desarrollar debe cumplir. El plan de pruebas se definir� antes de la implementaci�n y se emplear� para demostrar que el prototipo cumple el conjunto de especificaciones funcionales indicadas, que, l�gicamente, no tienen por qu� ser todas las que se imponen para el dise�o global del proyecto.

Por otro lado, una vez validado el prototipo, se proceder� a realizar una serie de pruebas de rendimiento en el sensor, con el objeto de poder evaluar la mejora de prestaciones que se obtiene en Ksensor gracias a la actualizaci�n. El plan de pruebas a seguir se definir� en el documento Estudio Comparativo de Rendimientos, en el cual se realizar� tambi�n un estudio comparativo a partir de los resultados obtenidos.

\section{Especificaciones de los entornos de aplicaci�n}
Los resultados de este proyecto deben estar enfocados para dar soluci�n a las necesidades que se plantean en el �mbito de la investigaci�n sobre la mejora de rendimientos en el an�lisis de tr�fico eficiente en redes de datos. En este sentido, los entornos de aplicaci�n que se plantean son los siguientes:

\begin{enumerate}
\item Entorno de desarrollo.
\item Entorno de pruebas.
\item Entorno de an�lisis comparativo.
\end{enumerate}

A continuaci�n se desarrollar�n las especificaciones particulares para cada uno de
estos entornos.

\subsection{Entorno de desarrollo}
El prototipo de Ksensor que se va a desarrollar en este proyecto se ejecutar� tambi�n en el nivel de kernel, con lo cual se deben considerar las particularidades, restricciones y complejidades que conlleva una implementaci�n de este tipo. En el kernel dif�cilmente pueden utilizarse herramientas de depuraci�n tipo gdb, y adem�s, un simple error o bug puede llegar a provocar el colapso total del sistema.

En consecuencia, ser� necesario reiniciar la m�quina, pero lo que es a�n peor, se destruir�n la mayor parte de los indicios o evidencias que pod�an apuntar el origen de dicho fallo. Es por ello que, en el desarrollo de Ksensor, se deben habilitar mecanismos de depuraci�n que permitan comprobar el funcionamiento de los m�dulos, adem�s de las facilidades de depuraci�n ya implementadas en el kernel, o bien en base a �stas.

A continuaci�n se describe el escenario en el que se desarrollar� el Ksensor. Se utilizar�n tres ordenadores, conectados entre s� a trav�s de una red independiente:
\begin{description}
\item[Gestor] Esta m�quina se utilizar� como estaci�n de trabajo, desde la cual se acceder� al resto de m�quinas de forma remota (v�a ssh).
\item[Sonda] Esta m�quina se utilizar� para la programaci�n, integraci�n y ejecuci�n de Ksensor. El sistema operativo ser� Debian GNU/Linux, en la versi�n del kernel 3.6.
\item[Inyector] Esta m�quina se utilizar� como inyector de tr�fico en las pruebas de validaci�n del prototipo, as� como en las pruebas de rendimiento.
\end{description}

La sonda y el inyector estar�n conectados entre s� a trav�s de un switch, en el que s�lo estar�n conectadas estas dos m�quinas, para poder controlar el tr�fico que se introduce en la red de captura. Ambas m�quinas se gestionar�n desde el gestor, utilizando una interfaz de red distinta. Se utilizar� esta configuraci�n para llevar a cabo el plan de pruebas de validaci�n definido en el Pliego de Condiciones.

La sonda se configurar� de tal manera que, cuando se produzca un error, se saque en los ficheros de registros \textit{logs} el mensaje correspondiente a trav�s del kernel. Por defecto, el mensaje se imprime en la pantalla y se vuelca a un fichero de \textit{logs}, gracias a la aplicaci�n syslogd.

No obstante, si el error es grave y el sistema colapsa (la m�quina deja de responder) los mensajes no se almacenar�n, con lo cual se pierde el motivo por el que se produjo el error. Para evitar esto, se podr�a utilizar el mecanismo netconsole implementado en Linux, mediante el cual se env�an los mensajes por medio de paquetes UDP, antes de que el sistema se cuelgue. La estaci�n de trabajo (gestor) ser�a el receptor o servidor de dichos mensajes, mientras que la sonda ser�a el cliente.

Otro de los mecanismos que puede ser implementado, aunque con algo m�s de tiempo dedicado, ser�a la posibilidad ejecutar un kernel de soporte para que en caso de fallo, tomara el control de la memoria, y guardara el estado completo de la memoria, para posteriormente ser enviado al desarrollador.

As� pues, las especificaciones a tener en cuenta en el desarrollo de la sonda son las siguientes:
\begin{itemize}
\item El kernel (Linux 3.6) se debe configurar con soporte para depuraci�n.
\item El prototipo de la sonda debe disponer de los mecanismos de depuraci�n necesarios para poder evaluar el funcionamiento del sistema con distinto nivel de detalle. En este sentido, se debe poder comprobar el funcionamiento global de la sonda, el de un m�dulo en concreto o el de una �nica funci�n.
\item El sistema debe ser capaz de enviar los mensajes de error que se produzcan a�n en el caso de que la m�quina se vaya a quedar bloqueada hasta que se ejecute un reinicio manual.
\item Se deben implementar funciones que permitan realizar pruebas unitarias sobre los m�dulos de la sonda, como por ejemplo, comprobar que se lee la l�gica de decisi�n de forma adecuada, que se desensamblan los paquetes correctamente, etc.
\item La sonda debe exportar diversas estad�sticas mientras se ejecuta, de modo que se pueda evaluar su comportamiento en determinadas circunstancias, por ejemplo, la forma en que se activa y desactiva el control de congesti�n.
\end{itemize}

\subsection{Entorno de pruebas}
El entorno de pruebas se refiere a las pruebas de rendimiento que se van a realizar en el laboratorio, con el objeto de valorar de forma cuantitativa la mejora de prestaciones que se obtiene con Ksensor en comparaci�n con la anterior implementaci�n del mismo. Para ello, se utilizar� la arquitectura para la automatizaci�n de pruebas dise�ada en \cite{AABS05}, adecu�ndola a las exigencias y caracter�sticas de nuestro sistema de captura y an�lisis de tr�fico de red.

El escenario en el que se realizar�n las pruebas ser� similar al descrito en la secci�n anterior. La sonda (Ksensor) y el inyector se conectar�n por medio de un switch, de tal forma que todo el tr�fico introducido en la red por el inyector ir� a parar a la sonda, el cual aplicar� la carga de an�lisis oportuna sobre los paquetes que captura.

Puesto que la finalidad principal de estas pruebas es obtener las caracter�sticas de rendimiento de las sondas, lo m�s interesante es comprobar el throughput de los mismos cuando se satura el enlace. Antes, se utilizaban varias m�quinas para saturar el enlace, ya que es muy dif�cil que un sistema operativo convencional pueda generar tal volumen de tr�fico.

Ahora, se dispone de inyectores hardware \cite{DAPR10} que tienen la capacidad de saturar un enlace a 1Gbps. Este tipo de inyectores no requieren otros equipos, por lo que se podr�a utilizar un enlace directo. En la arquitectura anterior se utilizaba un switch como forma de agregar tr�fico ethernet, pero se ha decidido no suprimir el switch para poder seguir disponiendo de unas estad�sticas externas.

As� pues, en esta configuraci�n se conectar�n dos m�quinas al switch: una para el sensor y una inyectora. En cierto modo, el inyector es el propio switch, por lo que en las pruebas de rendimiento ser� necesario obtener tambi�n las estad�sticas relativas a �ste, por llevar una contabilidad exacta por un medio externo.

Una vez planteado el escenario en el que se realizar�n las pruebas de rendimiento, se pueden definir ya las especificaciones que se deben tener en consideraci�n:
\begin{itemize}
\item La arquitectura de pruebas del sensor se basar� en la plataforma gen�rica
para la automatizaci�n de pruebas dise�ada en [BEAU05].
\item El sistema ha de ser capaz de lanzar bater�as de pruebas, configurando de
forma automatizada todos los agentes que participan en la prueba.
\item Se desarrollar� un agente para facilitar el encendido y apagado de ksensor
de forma automatizada.
\item Se proveer�n las utilidades necesarias para configurar los par�metros del
sensor y del inyector, as� como para obtener la informaci�n relativa a todos
los agentes que participan en la comunicaci�n.
\begin{itemize}
\item Par�metros de funcionamiento del sensor.
\begin{itemize}
\item Modos de captura.
\item Par�metros de la cola de paquetes.
\end{itemize}
\item Par�metros de funcionamiento del inyector
\begin{itemize}
\item Tasa de inyecci�n.
\end{itemize}
\item Estad�sticas de las interfaces de red.
\item Estad�sticas del switch.
\end{itemize}
\item Se deben dise�ar plantillas de pruebas para facilitar la realizaci�n de prue-
bas de rendimiento en laboratorio. Mediante estas plantillas se indicar�n los
par�metros con los que se van a ejecutar los agentes.
\end{itemize}

\subsection{Entorno de an�lisis comparativo}
A partir de los resultados obtenidos en las pruebas de rendimiento se puede realizar un estudio comparativo, con el fin de evaluar la mejora de prestaciones de ksensor. Para ello, se hace necesario definir primeramente un plan de las pruebas que se deben realizar, ya que de su precisi�n y exigencia depender� en gran medida el rigor de las conclusiones que se vayan a obtener. El plan de pruebas se detalla en la secci�n Plan de Pruebas en el documento Estudio Comparativo de Rendimientos, pero en este apartado se aportar�n una serie de especificaciones iniciales de las mismas.

Las pruebas de rendimiento que se realicen deben permitir cuantificar el throughput (tasa de paquetes procesados) al ser sometidos a diferentes tasas de inyecci�n, especialmente para cargas de tr�fico elevadas, con el objeto de poder conocer su comportamiento en redes de alta velocidad. Las pruebas deben facilitar, asimismo, conocer los par�metros de configuraci�n �ptimos en las diferentes versiones de ksensor. A partir de los resultados que se obtengan, se podr�n realizar gr�ficas comparativas, utilizando las herramientas necesarias para dicha tarea.

Por tanto, las especificaciones que se han de tener en cuenta al realizar el an�lisis comparativo entre las diferentes versiones son:
\begin{itemize}
\item Se deben identificar los par�metros susceptibles de estudio.
\item Se deben definir los prototipos de ensayo, a fin de realizar:
\begin{itemize}
\item Comparativa de las diferentes versiones de ksensor.
\item Comparativa de las diferencias entre los controles de congesti�n.
\item Comparativa de rendimiento para diferente n�mero de procesadores.
\end{itemize}
\item Juegos de ensayo: Para poder realizar un an�lisis comparativo de los diferentes prototipos y de las diferentes configuraciones, es necesario definir, en cada caso, la colecci�n de pruebas a realizar en el laboratorio.
\item Se deben proveer las utilidades necesarias para poder dibujar gr�ficas comparativas a partir de los resultados obtenidos en laboratorio.
\end{itemize}

\subsection{Especificaciones funcionales del sistema}
En esta secci�n se pretende ofrecer una descripci�n de las especificaciones o requerimientos que deben fijarse para el dise�o del sistema, de modo que el cumplimiento de �stas conduzca a la consecuci�n de los objetivos planteados al comienzo del presente documento.

Los m�dulos que se relatan tienen val�a por si mismos, ya que no son de uso exclusivo junto con Ksensor, y por lo tanto, los requerimientos ir�n orientados hacia un uso m�s general que Ksensor. A continuaci�n, se desglosar�n las caracter�sticas m�s importantes de todos los m�dulos nuevos y se mencionar�n las especificaciones de cada uno de ellos:
\begin{itemize}
\item M�dulo estad�sticas.
\item M�dulo traceo.
\end{itemize}

\subsection{M�dulo de estad�sticas}
El m�dulo de estad�sticas de que se va a implementar debe hacer mediciones sobre los par�metros que se considerar�n a continuaci�n. Tambi�n deber� proveer una interfaz sencilla de acceso a los m�smos, permitiendo la recogida de datos despu�s de cada prueba.

Los datos relevantes a ser estudiados deber�an ser:
\begin{itemize}
\item Los tiempos totales empleados en:
\begin{itemize}
\item Rutinas de servicio a la interrupci�n de captura de red.
\item Captura de paquetes por interfaz.
\item Capturar cada paquete.
\end{itemize}
\item N�meros absolutos de:
\begin{itemize}
\item Rutinas de servicio a la interrupci�n de captura de red.
\item Capturas de paquetes por cada interfaz.
\item Paquetes procesados para su captura.
\end{itemize}
\end{itemize}


\subsection{M�dulo de traceo}
La finalidad del m�dulo de traceo es, como su propio nombre indica, llevar a cabo trazas. Debe ser hecho de una manera simple, extensible y modular. La idea detr�s de este m�dulo es que se pueda ampliar para posibles necesidades de trazas que puedan surgir. Incluso, si en alg�n momento fuera necesario, a�adir varias trazas diferentes utilizando el mismo sistema.

La existencia de este m�dulo se propone como manera de comprobar modelos te�ricos de como los paquetes son recibidos en el sistema, pero pueden ser utilizados con muchos otros fines.

Este m�dulo deber� cumplir las siguientes premisas:
\begin{itemize}
\item Deber� ser lo m�s simple posible.
\item Tiene que ser r�pido y eficiente, sin gastar mucho tiempo, ya que en rutinas prioritarias podr�a tener un impacto grave.
\item Tener una interfaz f�cil de acceder sin necesidad de crear un mapa de memoria personalizado.
\end{itemize}
% !TeX spellcheck = es_ES
% !TeX root = main.tex

\chapter{Gesti�n}

% !TeX spellcheck = es_ES
% !TeX root = main.tex

\section{Riesgos}

% !TeX spellcheck = es_ES
% !TeX root = main.tex

\section{Referencias}


\pagenumbering{Roman}

\bibliography{bibliography}

\printindex

\end{document}