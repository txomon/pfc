% !TeX spellcheck = es_ES
% !TeX root = main.tex

\chapter{Riesgos}

El kernel de Linux est� pensado para ser de prop�sito general, y como tal, las rutinas de servicio a la interrupci�n est�n preparadas para unos vol�menes de tr�fico inferiores al �mbito de aplicaci�n de este proyecto.

Por ello, es importante considerar que aunque el desarrollo de kernel haya avanzado tanto en estos �ltimos a�os, es posible que los avances puedan suponer alg�n tipo de impedimento respecto al anterior kernel, a�n cuando se hayan observado mejoras generales en el funcionamiento de la nueva versi�n.

Debe tenerse en cuenta, que tanto como la estructura del kernel se ha vuelto m�s compleja, tambi�n puede estar convergiendo hacia una mejor respuesta hacia el usuario. De todos modos, este tipo de cambios aunque no son frecuentes, es un potencial riesgo del proyecto.

Otro de los riesgos de este proyecto es la posibilidad de que el servidor se sature y evite que el administrador se pueda poner en contacto con �l. Por lo tanto, aunque no se puede evitar que haya momentos en los que el equipo no responda, s� que se evitara en la medida de lo posible.